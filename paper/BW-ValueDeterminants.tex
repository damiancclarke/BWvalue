\documentclass[a4paper, 11pt]{article}

\usepackage{amsfonts}
\usepackage{amsmath}
\usepackage{amsthm}
\usepackage{appendix}
\usepackage{bm}
\usepackage{booktabs}
\usepackage[usenames, dvipsnames]{color}
\usepackage{graphicx}
\usepackage{epstopdf}
\epstopdfsetup{update}
\usepackage{helvet}
\usepackage{hyperref}
\usepackage{indentfirst}
\usepackage{lscape}
\usepackage{longtable}
\usepackage{morefloats}
\usepackage{natbib}
%\bibliographystyle{ecta}
\bibliographystyle{abbrvnat}\bibpunct{(}{)}{;}{a}{,}{,}
\usepackage{setspace}
\usepackage{subcaption}
\usepackage[capposition=top]{floatrow}
\usepackage{subfloat}
\usepackage[latin1]{inputenc}
\usepackage{tikz}
\usepackage{eurosym}
%\usepackage[pdf]{pstricks}

\usetikzlibrary{trees}
\usetikzlibrary{decorations.markings}


%MARGINS
 \topmargin   =  0.0in
 \headheight  =  -0.3in
 \headsep     =  0.7in
 \oddsidemargin= 0.0in
 \evensidemargin=0.0in
 \textheight  =  9.0in
 \textwidth   =  6.2in
% \setlength{\parindent}{4em}
 \setlength{\parskip}{1em}

\hypersetup{
    colorlinks=true,
    linkcolor=BlueViolet,
    citecolor=BlueViolet,
    filecolor=BlueViolet,
    urlcolor=BlueViolet
}

\title{On the Social and Private Value of Birth Weight}
\author{\small{Damian Clarke} \\ \small{Universidad de Santiago de Chile} \and \small{Sonia Oreffice} \\ \small{University of Surrey \& IZA}  \and \small{Climent Quintana-Domeque} \\ \small{University of Oxford \& IZA}}

\date{\today}
\begin{document}
\begin{spacing}{1.4}
\maketitle

\begin{abstract}
  Significant work exists documenting the educational and labour market returns
  to birth weight.  This is reflected in investments in a range of public
  programs targeting birth weight and early life health measures.  However,
  there is scarce evidence on the parental (private) valuation of birth
  weight.  In this paper we document that there exists a significant non-linear
  willingness to pay for birth weight even in the ``normal'' range of
  2,500 to 4,000 grams.  Using a series of discrete choice experiments, we
  estimate that parents are, on average, willing to pay \$1.72 for each
  additional gram of their babies' birth weight over the normal range.  The
  marginal willingness to pay (WTP) is particularly high at low birth weights,
  and turns negative at higher weights.  We document that the private WTP is of
  a similar size[???] to the inferred \emph{public} WTP based on large programs such
  as WIC.
\end{abstract}
\emph{JEL Classification Codes}: H31, I10, J01, J13, C90.\\
\emph{Keywords}: Birth weight; Infant health; Willingness to Pay; Discrete
Choice Experiments.

\newpage
\section{Introduction}
%-------------------------------------------------------------------------------

\citet{Royer2009}: ``Consistent with previous studies, I estimate a statistically significant relationship
between birth weight and long-run and intergenerational outcomes. In particular,
the heavier twin obtains more education, gives birth to heavier children, and has
fewer pregnancy complications. In sharp contrast to earlier research, however, these
effects tend to be quite small with the exception of pregnancy complications. For a
200 gram increase in birth weight, which likely is an achievable policy manipula-
tion, education would be projected to rise by roughly 0.04 of one year. These nega-
tive effects of birth weight do not appear to be persistent across generations, as the
estimated intergenerational correlation in birth weight is only 0.07. In contrast to
other studies (e.g., Black, Devereux, and Salvanes 2007), I find that the effects of
birth weight on long-run outcomes are nonlinear and for educational attainment, in
particular, are largest above 2,500 grams, the cutoff for defining low birth weight.
These findings suggest that babies with birth weights outside the lower tail of the dis-
tribution (i.e., outside the range of low birth weight) should receive more attention.''

\citet{BehrmanRosenzweig2004}: ``Our estimates provide a number of clear results. First,
they indicate that increasing fetal growth has a significant positive effect on schooling
attainment that is underestimated by 50\% if there is no control for genetic and family
background endowments as in cross-sectional estimates.
Second, our estimates indicate that intrauterine nutrient
consumption does not have any persistent effects on adult
BMI---increasing birthweight is not a cause of adult obesity.
Third, our results indicate that the heritable component of
birthweight plays the dominant role in the intergenerational
correlation of birthweights. Fourth, the estimates indicate
that intrauterine nutrient intake significantly affects adult
height, consistent with the literature that makes use of
height statistics to gauge childhood nutritional investments
over time and across countries. Fifth, although we
find evidence that augmenting birthweight, particularly
among lower-birthweight babies, would have significant
labor-market payoffs, we show that the strong cross-country
correlation between incomes and birthweight substantially
overstates the reduction in world earnings inequality that
would arise from reducing cross-country disparities in birth-weights.''

\citet{Blacketal2007}: ``We find that birth weight does matter. Consistent with earlier
work, we find that twin fixed effects estimates of the effect of birth weight on
short-run outcomes such as one-year infant mortality are much smaller than their
cross-sectional equivalents. However, studying only short-run outcomes may lead to
incorrect inferences about the longer-run effects of birth weight; we find
that birth weight has a significant effect on longer-run outcomes
such as height, IQ at age 18, earnings, and education,
and the fixed effects estimates are similar in size to cross-sectional ones.''

\citet{Oreopoulosetal2008}: ``We find evidence of longer-term
consequences of infant health both across families, within siblings, and
within twin pairs, although different measures of infant health predict outcomes
differently. The
results suggest strong effects of infant health on death between ages one and 17,
grade completion, and months on social assistance after age 18, even for ranges
not considered overtly concerning (for example, birth weights between 2,500 and
3,500 grams and Apgar scores of seven or eight)''

\citet{Figlioetal2014} We make use of a new data resource -- merged birth and school records for all children born in Florida from 1992 to 2002 -- to study the relationship between birth weight and cognitive development. Using singletons as well as twin and sibling fixed effects models, we find that the effects of early health on cognitive development are essentially constant through the school career; that these effects are similar across a wide range of family backgrounds; and that they are invariant to measures of school quality. We conclude that the effects of early health on adult outcomes are therefore set very early. 

\citet{ConleyBennet2000}: ``The results, presented in Table 2, indicate
that low birth weight negatively affects educational progress, even
after factoring out family-specific conditions.''

Citations to add: \citet{CookFletcher2015,Bharadwajetal2015,Fletcher2011}

\section{Data}
We collected data on preferences over birth characteristics by running a
discrete choice experiment on Amazon's Mechanical Turk (MTurk) online platform.
This platform is a market place which provides access to a pool of US MTurk
workers (survey respondents) who are paid per completed ``HIT'' (Human
Intelligence Task).  We requested that a sample of respondents complete a
discrete choice experiment (described further below) as well as a series of
demographic questions.  These demographic questions were asked after the
completion of the survey to avoid any framing or experimenter demand effects.
Mechanical Turk respondents have been documented to have a series of desirable
characteristics, and be more representative of the US population than other
frequently-used subject pools such as college student samples
\citep{Berinskyetal2012}.  Mechanical Turk samples are increasingly used both
within and outside of the economic literature (see for example
\citet{Kuziemkoetal2015,Jordanetal2016}).

We recruited a sample of 1,002 respondents aged 18 or over, and conducted
the experiment on Monday September 19\textsuperscript{th}, 2016.  Workers
were paid \$1.10 for a 6 minute experiment (average length), resulting in
an effective hourly pay rate of approximately \$11.  We required that
respondents must be from the United States\footnote{Workers on Mechanical
  Turk require a United States Social Security Number.}, and in order to
maximise the likelihood that workers were based in the US at the time of
completing the survey, the survey was launched at 9:00 AM EST.  By 2:13 PM
of the same day 1,002 valid responses were collected.  We also required
that workers had completed at least 100 tasks on MTurk in the past, and
had achieved an approval rating of greater than 95\% on these tasks.
These restrictions have been used in the past in Mechanical Turk research
\citep{Berinskyetal2012,Francis-TanMialon2015}.  Of the 1,002 completed
respondents, we removed a small number based on a series of pre-defined
consistency checks.  These were: (a) workers whose geographical IP address
placed them outside of the US at the time of survey (36 respondents, or
3.6\%), any respondents who failed a consistency check where a question was
repeated at the beginning and end of the demographic portion of the survey
(8 respondents), and any respondents completing the entire exercise in
under 2 minutes (6 respondents).\footnote{In the appendix to this paper we
  demonstrate that nonetheless, our results remain largely unchanged if we
  do not remove these respondents.}  The final sample thus consists of 952
respondents.

The geographic location of these respondents within the United States (based
on their IP address) is provided in figure \ref{geography}.  The geographic
coverage is broadly representative of the US population.  In appendix table
\ref{tab:cover} we compare our MTurk respondent covarage with the US population
from 2015 \citep{CensusBureau2015}.  In general we see that the MTurk sample
lines up well with the national population at the state level, however there
are a number of exceptions, such as the lower number of respondents from
California.  In our principal specifications we always use the unweighted
MTurk sample, however as an alternative test we also re-weight the sample
to balance state and demographic characteristics, as suggested by
\citet{Francis-TanMialon2015}, and discussed further in the following section.

Finally, summary statistics of the respondents are provided in table
\ref{sumstats}.  Slightly more than half of respondents are female (56\%),
and the ages of respondents range from 18 to 75 years (with a mean age of 36
years).  Approximately half of respondents are parents (51\%).  DISCUSS
HERE COMPARISON WITH USA POPULATION.  SHOULD WE USE ACS TO COMPARE?
%\subsection{Birth Certificate Data}
%\subsection{Mechanical Turk Survey Data}


\section{Methodology}
\label{scn:methods}
%\subsection{Observable Determinants of Birth Weight}
%\subsection{Eliciting Willingness to Pay in a Discrete Choice Experiment}
In order to estimate the perceived importance of birth weight, and the willingess
to pay for additional weight, we run a Discrete Choice Experiment (DCE) on a
large sample of US-based respondents. A DCE is a type of Conjoint Analysis (CA):
an experiment in which respondents are asked to choose their preferred option
from a set when a number of attributes are varied simultaneously.
CA was borne from early work in consumer theory in which tastes for goods owe
to the collection of their characteristics \citep{Lancaster1966}.  In the
past CA has been used to measure preferences over medical care in a variety of
contexts, including the valuation of waiting times \citep{Propper1990,Propper1995},
alternative miscarriage treatment options \citep{RyanHughes1997}, asthma
medications \citep{Kingetal2007}, or depression management \citep{Wittinketal2010}.
Conjoint analysis has recently been shown to be the best-performing experimental
design when compared with actual choices made \citep{Hainmuelleretal2015}.

The birth choice experiment consists of asking respondents to consider a series
of two birth outcomes, while focusing on four attributes of each birth.  These
attributes are the sex of the child (Boy or Girl), the out of pocket expenses
associated with the birth (\$250, \$750, \$1000, \$2000, \$3000, \$4000, \$5000,
\$6000, \$7500, or \$10000), the baby's weight at birth (5lbs, 8oz; 5lbs, 13oz;
6lbs, 3oz; 6lbs, 8oz; 6lbs, 13oz; 7lbs, 3oz; 7lbs, 8oz; 7lbs, 13oz; 8lbs, 3oz;
8lbs, 8oz; or 8lbs, 13oz), and the season in which the baby is born (Winter,
Spring, Summer or Fall).  These attributes are all orthogonally varied, implying
that the effect of each characteristic on the likelihood that a particular birth
is chosen is separately identified \citep{Marshalletal2010}). Each respondent 
was asked to consider 7 pairs of birth outcomes.  In each case the two pairs
were displayed side-by-side on a single screen, and respondents were asked to
indicate which was their preferred outcome.  As well as randomising the level
of each attribute on each profile, the order of the attributes was randomised,
however to reduce the cognitive load to respondents the ordering of attributes
was only randomised once, and then fixed across the seven pairings that the
respondent ranked.

The levels of attributes were chosen to represent plausbile values from the
US population \citep{RyanFarrar2000}, and extreme values were avoided to
prevent the likelihood of ``grounding effects'' (or corner solutions).  In
order to minimise the likelihood that respondents would employ simple
heuristics in answers, we limited the number of attributes (4) which need be
considered.  Birth weights were always presented in pounds and ounces, given
that this experiment was entirely run with a US-sample.  As well as indicating
that all births were complication-free, only birth weights over the normal
range of 2,500--4,000 grams were included (11 evenly-spaced weights were
defined in this range, and expressed in pounds and ounces in the experiment).%
\footnote{This range includes the majority of all births in the US.  According
  to full vital statistics from
  2013 (from the National Vital Statistics System), 15.91\% of births occurred
  with weights outside of this range (refer to appendix figure \ref{bwt-nvss}.)
  Of these, 8.02\% were low birth weight ($<$ 2,500 grams), and 7.89\% were
  large for gestational age at birth ($>$ 4,000 grams).
} 
An opt-out option was not included in any of the discrete choices.  This
has been suggested to have desired properties such as avoiding non-random
opt-out of all questions \citep{Veldwijketal2014,BekkerGrobetal2012}.  The
DCE's framing and the explanation of the attributes shown to respondents is
displayed in appendix figures \ref{DCE-frame} and \ref{DCE-options}.  In
appendix figure \ref{DCE-example} we display a single choice scenario as
presented to respondents.  Appendix \ref{app:procedure} describes the
survey procedure followed by respondents.

Consider a sample of $i\in \{1,\ldots,N\}$ individuals, each of whom consider
$K$ choice tasks in which they must decide between $J$ options (or profiles).
Each profile contains $L$ attributes, where each particular attribute $l$
consists of discrete levels of the variable.  In the case of the DCE
described above, we have $N=952$ respondents, $K=7$ choice tasks per
respondent, $J=2$ profiles per task, and $L=4$ attributes.\footnote{These
  4 attributes have 2, 10, 11 and 4 levels respectively for sex, out of pocket
  costs, birth weight and season of birth.} We follow
\citet{Hainmuelleretal2013} in defining a treatment vector $T_{ijk}$.
This treatment vector has $L$ cells, and summarizes for individual $i$,
at choise task $k$, for profile $j$, the full set of attributes observed.
Each particular attribute $T_{ijkl}$ is randomly assigned from among all
the levels of $l$, the assignation of which is orthogonal to all other
attributes the respondent sees.  Using the potential outcomes framework,
we define a binary variable $Y_{ijk}(\bar{\mathbf{t}})$ which takes the
value 1 if respondent $i$ would choose profile $j$ on choice set $k$ if
faced with the set of attributes $\bar{\mathbf{t}}$, or 0 if the profile
would not be chosen.

We are interested in estimating two quantities.  Firstly, we would like to
estimate, \emph{ceteris paribus} the likelihood that a birth is chosen given
that a particular birth weight is observed (compared with an omitted base
category).  Secondly, we would like to estimate the willingness to pay for
season of birth, by combining the information from both variations in birth
weight and variations in out-of-pocket costs.

\citet{Hainmuelleretal2013} call this first quantity the Average Marginal
Component Effect (AMCE) and demonstrate that under reasonably weak
assumptions\footnote{These assumptions relate to randomization of attributes,
  and stability of respondent behaviour regardless of the number of profiles
  that they have seen or the order of the attribute in the profile.  This
  first assumption holds by construction in our experiment.  In the following
  section we return to explicitly test for violations of the latter two
  assumptions.}, it can be recovered using a non-parametric subclassification
estimator, conditional regression, or a simple difference of means.  The
logic of the AMCE\footnote{Formally, the AMCE is defined as
  \citep{Hainmuelleretal2013}:
  \[
  E[Y_i(t_1,T_{ijk[-l]},\mathbf{T}_{i[-j]k})-Y_i(t_0,T_{ijk[-l]},\mathbf{T}_{i[-j]k})|(T_{ijk[-l]},\mathbf{T}_{i[-j]k})\in\tilde{\mathcal{T}}]
  \]
  which can be quite easily calculated by integrating over all of the other
  attributes and levels except for $t_1$ (the treatment of interest) and $t_0$
  (the basline level for the attribute). These other attributes and levels are
  denoted as the set $\tilde{\mathcal{T}}$ here.} is to capture the change in
the likelihood that a given profile would be chosen if the
$l$\textsuperscript{th} component were changed from $t_0$ to $t_1$, or in our
case, a change in birth weight.

Under the controlled randomization in conjoint analysis, \citet{Holland1986}'s
fundamental problem of causal inference is resolved by construction, as on
average there will be no correlation between observing the particular level
of an attribute and individual correlates. Treatment units are thus those who
observe a particular $t_1$, while those who do not act as controls.  In
practice to estimate the change in likelihood that a particular birth weight
(or any attribute) changes the likelihood that a birth is chosen, we estimate
the following regression:
\begin{equation}
Pr(Y_{ijk}=1) = \alpha + \beta Costs + \sum_{l=2}^{11} \gamma_l BW_l + \sum_{l=2}^{4} \delta_l SOB_l +  \kappa Girl + \mu_j + \phi_k + \varepsilon_{ijk}.
\end{equation}
Here we can include option and profile order fixed effects ($\mu_j$ and
$\phi_k$ respectively), and standard errors are clustered at the level
of the respondent to capture the (likely) positive correlations among
choices based on attributes by a particular respondent.\footnote{This
  will capture, for example, that a respondent who is encouraged to choose
  a particular birth given a higher birth weight being similarly likely to
  choose other birth options when observing higher birth weights.} The
coefficients $\gamma_l \ \forall\ l$ are the principal effects of interest,
and capture the likelihood that a birth is chosen given a particular birth
weight.  We omit from the regression the lowest birth weight category as the
baseline level, implying that all coefficients should interpreted as the
marginal likelihood of choosing a birth given birth weight $l$ in place of
the lowest birth weight (2,500 grams).

In the above model, we are also able to estimate willingness to pay for a
marginal change in birth weight.  Consider the two AMCEs, $\beta$ and $\gamma_2$
(where $\gamma_2$ is chosen without loss of generality).  These coefficients
represent the marginal effect on the likelihood of choosing a particular birth
given an increase in the particular attribute, conditional on all other
attributes:
\[
\beta=\frac{\partial Pr(Y_{ijk}=1)}{\partial Costs} \qquad \gamma_l=\frac{\partial Pr(Y_{ijk}=1)}{\partial BW_2}.
\]
The marginal rate of substitution between the particular birth weight $BW_2$
and the price of a given birth (the out of pocket costs) thus gives the change
in costs that an average respondent would willing to withstand for a marginal
increase in birth weight:
\[
MRS_{BW,Costs}=\frac{\frac{\partial Pr(Y_{ijk}=1)}{\partial BW_2}}{\frac{\partial Pr(Y_{ijk}=1)}{\partial Costs}}.
\]
The quantity is precisely the willingness to pay, ie, the change in financial
resources that makes a resondent indifferent between the higher or lower birth
weight:
\[
WTP_{BW}=-\frac{\gamma_2}{\beta}=-\frac{\partial Cost}{\partial BW_2}.
\]
Note that in the above calculation we take the negative so that costs are
interpreted as the (positive) amount that must be paid rather than the
(negative) change in financial resources.  This $WTP_{BW}$ can also be
derived quite straightforwardly from a model of the indirect utility
function as described in \citet{Zweifeletal2009}, and applied in
\citet{Clarkeetal2016}.

Unlike the calculation for the ACME for birth weight, the above WTP
is not associated with a standard error and confidence interval.  In
order to calculate the confidence intervals associated with the WTP
we use the delta method, which is both simple and shown to perform
well under simulation \citep{Hole2007}.  We also find that these
confidence intervals are quite comparable to those produced when using
block bootstrapping.

\section{Results}
\subsection{The Experimental Valuation of Birth Weight}
In figure \ref{DCE-samp} we present experimental results.  This figure
displays point estimates of the likelihood of preferring a particular
birth given each characteristic, compared with an omitted base category
for each characteristic.  Along with each point estimate, the 95\%
confidence interval is plotted, clustering by individual.  In figure
\ref{DCE-samp} we present cost as a linear variable measured in 1000s of
dollars.  In appendix figure \ref{DCE-full-samp} the same results are
presented with costs presented as the same categorical measure displayed
to respondents.

The top panel displays the likelihood of choosing a birth given a
particular birth weight, compared to being shown the minimum sample
birth weight of 5lbs, 8oz (2,500 grams).  In each case higher birth
weights are associated with a greater likelihood of choosing the
birth.  The most preferred birth weight (based on point estimates) is
7lbs, 8oz (3,400 grams), which results in a birth being approximately
18\% more likely to be chosen than the omitted base category.  The
magnitudes of the estimates are large.  With the exception of 5lbs, 13
oz, all higher birth weights are at least 10\% more likely to be chosen,
and in each case the differences is statistically significant.  What's
more, there appears to be a hump-shaped pattern, with the most
preferred births being those towards the middle of the (normal) birth
weight range, and lower preferences for those at the extremes.  We
return to this point below when discussing the estimates for the
willingness to pay for birth weight.

Briefly, turning to other characteristics included in the conjoint analysis,
we find no evidence of any elicited preference for the baby's gender
on average in the full sample.  Indeed, in both specifications displayed
in table \ref{WTPreg} estimated coefficients on the baby being a girl are
quite tightly estimated zeros (ranging from 0.000 to 0.001 with clustered
standard errors of 0.010). This finding echoes that in \citet{LhilaSimon2008}
who find no evidence of differential investments \emph{in utero} in boy and
girl children in the US when the gender is known.  However, when estimating
separately by the gender of the respondent, we do observe a preference for
boy children among male respondents (table \ref{WTPgreg}). This is in
agreement with the results of \citet{DahlMoretti2008} who document a demand
for sons, particularly among fathers.  When considering season of birth
we observe a greater likelihood to choose births in the spring, evidence
of a demand for certain seasons of birth \citep{Clarkeetal2016}. Finally,
and unexpectedly, we observe that all else equal, higher costs result in
a birth being less likely to be preferred.  On average, for each additional
\$1000 in out of pocket expenses, the likelihood of choosing a birth falls by
nearly 10\%.  The non-linear estimates of these parameters are displayed
in appendix figure \ref{DCE-full-samp}.

As discussed in section \ref{scn:methods}, we can combine estimates of
characteristics with those on out of pocket costs to generate estimates
of the willingness to pay for each characteristic.  In table \ref{WTPreg}
we present regression results which display the coefficient on birthweight
(assuming in column 1 a linear functional form).  By comparing the change
in likelihood of choosing a birth based on an increase in birth weight
with the change in likelihood due an increase in costs, we estimate that
the willingness to pay for an additional 1000 grams in the full sample
is \$1438.3, or \$1.44 per gram.  However, as we observe in column 2,
the relationship between birth weight and likelihood of choosing a birth
is non-linear.  In figure \ref{WTP-relative} we document the WTP of all
birth weight options, with respect to the minimum birth weight in the
sample.  We observe that the largest relative difference occurs at 3,400
grams, with a WTP of nearly \$3000USD, or \$2.14 per gram.

In table \ref{WTPgreg} we observe that the WTP is highest among parents,
at \$1.72 per gram, compared to \$1.17 among those who are not (currently)
parents.

DISCUSS VALUES AND THE WTP FOR WEIGHT DISPLAYED NON-PARAMETRICALLY IN
FIGURE \ref{WTP-relative} (WTP COMPARED WITH MINIMUM WEIGHT) AND FIGURE
\ref{WTP-marginal} (MARGINAL WTP COMPARED WITH SLIGHTLY LOWER WEIGHT).

SHOULD WE MENTION THE ``13 EFFECT''?
\paragraph{Population Re-weighting}

\subsection{Estimating WTP from Various Public Programs}


\section{Conclusion}

\newpage
\noindent\textbf{Notes}: This experiment documented in this paper has passed ethical approval at the Oxford Centre of Experimental Social Sciences (CESS), and been registered as project ETH-160128161.

\bibliography{./refs}

\clearpage
\section*{Figures and Tables}
\begin{figure}[htpb!]
  \begin{center}
    \caption{Geographic Coverage of Respondents}
    \label{geography}
  \includegraphics[scale=0.9]{../results/DCE/Summary/surveyCoverage.eps}
  \end{center}
  \floatfoot{\textsc{Notes:} The full survey sample consists of 1,002 respondents.  The final estimation sample consists of 952 respondents given that it removes respondents whose geographic IP suggested a non-US location (36 respondents, 3.6\%), those who failed to respond that their educational attainment was identical at the beginning and end of the survey (8 respondents, 0.8\%), and those who completed the discrete choice experiment in under two minutes (6 respondents, 0.6\%).}
\end{figure}

\begin{table}[htpb!]
  \begin{center}
    \caption{Summary Statistics of Experimental Respondents}
    \label{sumstats}
    \begin{tabular}{lccccc} \toprule
    \input{./../results/DCE/Summary/MTurkSum-clean.tex}
    \bottomrule
    \multicolumn{6}{p{11.4cm}}{{\footnotesize\textsc{Notes:} Refer to figure \ref{geography} for a discussion of the experimental sample. Years of education, total income and hourly MTurk earnings are calculated from categorical variables.}}
  \end{tabular} 
  \end{center}
\end{table}

\clearpage

\begin{figure}[htpb!]
  \begin{center}
    \caption{Discrete Choice Experimental Results (Main Sample)}
    \label{DCE-samp}
  \includegraphics[scale=0.9]{../results/DCE/Figures/Conjoint_Sample_continuous.eps}
  \end{center}
  \floatfoot{\textsc{Notes:} }
\end{figure}

\input{./../results/DCE/Regressions/conjointWTP.tex}

\begin{landscape}
\input{./../results/DCE/Regressions/conjointGroups.tex}
\end{landscape}

\begin{figure}[htpb!]
  \begin{center}
    \caption{Relative Willingness to Pay for Birthweight}
    \label{WTP-relative}
  \includegraphics[scale=0.74]{../results/DCE/Figures/WTP_relative.eps}
  \end{center}
  \floatfoot{\textsc{Notes:} Each point and confidence interval are with respects to the baseline (omimtted) category of 2,500 grams, the minimum displayed birthweight.  Willingness to pay is determined as the ratio between the particular birthweight and out of pocket costs estimated as average marginal effects in a logit regression.  95\% confidence intervals displayed are calculated using the delta method.}
\end{figure}

\begin{figure}[htpb!]
  \begin{center}
    \caption{Marginal Willingness to Pay for Birthweight}
    \label{WTP-marginal}
  \includegraphics[scale=0.74]{../results/DCE/Figures/WTP_marginal.eps}
  \end{center}
  \floatfoot{\textsc{Notes:} Each point and confidence interval compare the willingness to pay for a particular birthweight compared to the preceding birthweight.  Willingness to pay is determined as the ratio between the particular birthweight and out of pocket costs estimated as average marginal effects in a logit regression. 95\% confidence intervals displayed are calculated using the delta method.}
\end{figure}


\clearpage
\setcounter{table}{0}
\renewcommand{\thetable}{A\arabic{table}}
\setcounter{figure}{0}
\renewcommand{\thefigure}{A\arabic{figure}}

\appendix
\section{Appendix Figures and Tables}
\begin{figure}[htpb!]
  \begin{center}
    \caption{Birthweight from Administrative Data}
    \label{bwt-nvss}
  \includegraphics[scale=0.9]{../results/births/birthweight.eps}
  \end{center}
  \floatfoot{\textsc{Notes:} }
\end{figure}


\begin{figure}[htpb!]
  \begin{center}
    \caption{Discrete Choice Experimental Results with Categorical Costs (Main Sample)}
    \label{DCE-full-samp}
  \includegraphics[scale=0.9]{../results/DCE/Figures/Conjoint_Sample.eps}
  \end{center}
  \floatfoot{\textsc{Notes:} }
\end{figure}

\begin{figure}[htpb!]
  \begin{center}
    \caption{Discrete Choice Experimental Results (Full Sample)}
    \label{DCE-samp}
  \includegraphics[scale=0.9]{../results/DCE/Figures/Conjoint_All_continuous.eps}
  \end{center}
  \floatfoot{\textsc{Notes:} }
\end{figure}


%\begin{figure}[htpb!]
%  \begin{center}
%    \caption{Discrete Choice Experimental Results with Categorical Costs (Full Sample)}
%    \label{DCE-full-all}
%  \includegraphics[scale=0.9]{../results/DCE/Figures/Conjoint_All.eps}
%  \end{center}
%  \floatfoot{\textsc{Notes:} }
%\end{figure}

\input{./../results/DCE/Regressions/conjointGroups-wts.tex}




\begin{figure}[htpb!]
  \begin{center}
    \caption{Discrete Choice Experiment Framing}
    \label{DCE-frame}
  \includegraphics[scale=0.7]{surveyQs/Q1.png}
  \end{center}
  \floatfoot{\textsc{Notes:} }
\end{figure}

\begin{figure}[htpb!]
  \begin{center}
    \caption{Discrete Choice Experiment Options}
    \label{DCE-options}
  \includegraphics[scale=0.7]{surveyQs/Q2.png}
  \end{center}
  \floatfoot{\textsc{Notes:} }
\end{figure}

\begin{figure}[htpb!]
  \begin{center}
    \caption{Discrete Choice Experiment Example}
    \label{DCE-example}
  \includegraphics[scale=0.7]{surveyQs/Q3.png}
  \end{center}
  \floatfoot{\textsc{Notes:} }
\end{figure}


\begin{figure}[htpb!]
  \begin{center}
    \caption{Testing Assumption 2}
    \label{DCE-asm2}
  \includegraphics[scale=0.82]{../results/DCE/Figures/roundOrder.eps}
  \end{center}
  \floatfoot{\textsc{Notes:} }
\end{figure}

\begin{figure}[htpb!]
  \begin{center}
    \caption{Testing Assumption 3}
    \label{DCE-asm2}
  \includegraphics[scale=0.82]{../results/DCE/Figures/attributeOrder.eps}
  \end{center}
  \floatfoot{\textsc{Notes:} }
\end{figure}

\begin{landscape}
\begin{figure}[htpb!]
  \begin{center}
    \caption{Mechanical Turk Front Page}
    \label{MTurkAdd}
  \includegraphics[scale=0.65]{surveyQs/MTurkScreen.png}
  \end{center}
  \floatfoot{\textsc{Notes:} }
\end{figure}
\end{landscape}

\clearpage
\end{spacing}

\begingroup
\setlength{\LTleft}{-20cm plus -1fill}
\setlength{\LTright}{\LTleft}
\begin{longtable}{lccc} 
  \caption{Geographical Coverage} \label{tab:cover} \\
  \hline
  State Name & Percent & Percent & Difference \\
             & MTurk      & Census Bureau & (\%)          \\ \hline \endhead
  \input{./../results/DCE/Summary/GeographicCoverage.tex}
  \hline 
  \multicolumn{4}{p{10.4cm}}{{\footnotesize\textsc{Notes:} Columns present the percent of respondents from the MTurk sample, the percent of residents according to US Census Bureau records (2015), and the difference between the percent of MTurk respondents and residents.}}
\end{longtable}
\endgroup


\section{Survey Response Procedure}
\label{app:procedure}
Below we describe the survey response procedure as seen by survey respondents.
\begin{enumerate}
\item All respondents meeting survey criteria ($>95\%$ approval rating, $>100$ completed MTurk tasks, US based, and non-participants in the pilot) were able to see the Mechanical Turk HIT with the title ``Link to Survey'' along with the description displayed in appendix figure \ref{MTurkAdd}. Respondents are instructed that payment is conditional upon completing the survey and providing a randomized code which is displayed at the end of the survey.
\item Respondents accept participation and are directed to the discrete choice experiment on the Qualtrics survey platform.
\item Respondents must complete each question in order to move forward, and after completing the survey the randomized code is displayed.
\item Respondents return to the MTurk front page, enter their unique completed survey code and receive payment.
\end{enumerate}


\end{document}
