\documentclass[a4paper, 11pt]{article}

\usepackage{amsfonts}
\usepackage{amsmath}
\usepackage{amsthm}
\usepackage{appendix}
\usepackage{bm}
\usepackage{booktabs}
\usepackage[usenames, dvipsnames]{color}
\usepackage{graphicx}
\usepackage{epstopdf}
\epstopdfsetup{update}
\usepackage{helvet}
\usepackage{hyperref}
\usepackage{indentfirst}
\usepackage{lscape}
\usepackage{longtable}
\usepackage{pdflscape}
\usepackage{morefloats}
\usepackage{natbib}
\bibliographystyle{aea}
%\bibliographystyle{abbrvnat}\bibpunct{(}{)}{;}{a}{,}{,}
\usepackage{setspace}
\usepackage{subcaption}
\usepackage[capposition=top]{floatrow}
\usepackage{subfloat}
\usepackage[latin1]{inputenc}
\usepackage{tikz}
\usepackage{eurosym}
%\usepackage[pdf]{pstricks}

\usetikzlibrary{trees}
\usetikzlibrary{decorations.markings}


%MARGINS
 \topmargin   =  0.0in
 \headheight  =  -0.3in
 \headsep     =  0.7in
 \oddsidemargin= 0.0in
 \evensidemargin=0.0in
 \textheight  =  9.0in
 \textwidth   =  6.2in
% \setlength{\parindent}{4em}
 \setlength{\parskip}{1em}

\hypersetup{
    colorlinks=true,
    linkcolor=BlueViolet,
    citecolor=BlueViolet,
    filecolor=BlueViolet,
    urlcolor=BlueViolet
}

\title{On the Value of Birth Weight}
\author{\small{Damian Clarke} \\ \small{Universidad de Santiago de Chile} \and \small{Sonia Oreffice} \\ \small{University of Surrey \& IZA}  \and \small{Climent Quintana-Domeque} \\ \small{University of Oxford \& IZA}}

\date{\today}
\begin{document}
\begin{spacing}{1.4}
\maketitle

\begin{abstract}
  Significant work documents the educational and labour market returns
  to birth weight, which are reflected in investments in public
  programs targeting birth weight and early life health measures.  However,
  there is scarce evidence on the private valuation of birth
  weight. In this paper we document that there exists a significant and non-linear
  willingness to pay for birth weight in the ``normal'' range of
  2,500 to 4,000 grams.  Using a series of discrete choice experiments conducted
  on Amazon's \emph{Mechcanical Turk} platform, we
  estimate that parents are, on average, willing to pay \$1.72 for each
  additional gram of their babies' birth weight over the normal range.  The
  marginal willingness to pay (WTP) is particularly high at low birth weights,
  and turns negative at higher weights.  Nonetheless, back-of-the-envelope
  calculations suggest that the parental WTP for birth weight still falls short
  of the expected present value of birth weight on the labour market for a
  US-born child.
\end{abstract}
\emph{JEL Classification Codes}: H31, I10, J01, J13, C90.\\
\emph{Keywords}: Birth weight; Infant health; Willingness to Pay; Discrete
Choice Experiments.

\newpage
\section{Introduction}
%-------------------------------------------------------------------------------
A baby's birth weight is the most frequently measured and used endowment to
capture the immediate stock of health early in life \citep{AlmondCurrie2011,
Almondetal2017}.
The importance of the fetal period as a predictor of health throughout the life
course has been recognised in a series of influential papers by Barker and
coauthors on the fetal origins of disease \citep{Barkeretal1989,Barker1990,
  Barker1995}, with considerable and ever-growing evidence in economics that
insults to fetal
health have enduring and considerable costs throughout life \citep{Almond2006,
  CurrieMoretti2007,Caseetal2005,Almondetal2009,Blacketal2007}.
These findings justify considerable investments in programs targeted at
babies with poor endowments early in life, such as those focusing on low birth
weight babies \citep{Almondetal2005,Bharadwajetal2013} and pre-natal nutrition
programs such as the Special Supplemental Nutrition Program for Women Infants
and Children (WIC).\footnote{The total national spending on WIC in 2015 was 6.2
billion dollars.} There are also non-targeted programs which have been reported to impact birth outcomes.\footnote{\citet{Almondetal2011} investigate the impact of the introduction of the modern Food Stamp Program (FSP) on birth outcomes. Using variation in the month FSP began operating in each U.S. county, they find that pregnancies exposed to FSP three months prior to birth yielded deliveries with increased birth weight, with the largest gains at the lowest birth weights.}

Despite a large body of evidence on the importance of birth weight, and
considerable public investment, little is
known regarding the private valuation of this birth outcome. Two questions  arise: First, are taxpayers (in general) aware of the value of birth weight? Second, how do parents perceive the value of birth weight?
To the degree that a wide range of (costly) parental behaviors can
positively impact birth weights \citep{RosenzweigSchulz1983,
  ChevalierOSullivan2007,SextonHebel1984}, the perceived importance of
birth weights to parents may have significant effects on these behaviors,
and on birth weight and child outcomes throughout the life course. Thus,
knowing the value which people place on birth weight is a fundamental policy issue, and in particular a key ingredient
to policies focused on parental behavior prior to and during gestation.

In this paper we aim to provide the first estimates of the importance of
birth weight to individuals (and parents), and estimate the Willingness to Pay (WTP)
for birth weight.
In order to do so, we conduct a series of discrete choice experiments
on Amazon Mechanical Turk, an online labor market platform increasingly used both
within and outside of the economics literature \citep{Kuziemkoetal2015,
  Jordanetal2016}.  We conducted these experiments with approximately
1,000 respondents, who were asked to consider around 14,000
different birth profiles with a number
of different characteristics, each orthogonally varied both within and
between experimental subjects.  Specifically, we performed conjoint analysis,
a method first described by \citet{Lancaster1966}, which has recently
been shown to perform more favorably than experimental techniques when
compared with real-world choice behavior \citep{Hainmuelleretal2015}.

These experiments allow respondents to reveal their preferences (or lack
of preferences) over a range of birth characteristics.  In particular, we
randomize jointly over a baby's birth weight, birth timing, birth costs,
and gender.  While focusing here on the value of birth weight, these
experiments lead to a number of interesting findings related to the
demand for other previously documented birth outcomes such as the demand
for sons \citep{DahlMoretti2008} and the demand for season of birth
\citep{Clarkeetal2016}.  Throughout this paper
we focus on birth weight over the normal range of 2,500 to 4,000 grams.
While many studies focus on low birth weight (LBW, or weights less than
2,500 grams) as their indicator of interest, we restrict our analysis
to the ``normal'' range for a number of reasons.  Firstly, it has been
shown that continuous measures of birth weight actually have greater
explanatory power for a large range of variables than an LBW indicator
\citep{Blacketal2007}.  Secondly, recent evidence suggests that marginal
increases in birth weight within
this normal weight range are particularly important for well-being.  \citet{Royer2009}, for example, suggests that
babies born in the normal range of weights should receive \emph{more}
research attention.\footnote{In full, \citep{Royer2009} reports (p.\ 52):
  \begin{quote} ``I find that the effects of birth weight on long-run
    outcomes are nonlinear and for educational attainment, in particular,
    are largest above 2,500 grams, the cutoff for defining low birth weight.
    These findings suggest that babies with birth weights outside the
    lower tail of the distribution (i.e., outside the range of low birth
    weight) should receive more attention.''
  \end{quote}
}
  Finally, from a purely practical standpoint, we focused on the
normal range of birth weights to avoid priming effects (i.e., respondents linking low birth
weight with other health conditions), thus confounding our estimates
for the WTP of birth weight alone.

Our results suggest that, similar to researchers and policymakers,
individuals (and parents) view birth weight as an important and valuable characteristic.
All else constant, a baby weighing 3,400 grams (7lbs, 8oz) is 18 percentage
points (pp) more likely to be chosen than one weighing 2,500 grams (5lbs, 8oz).
We estimate that over the normal range of birth weights examined,
experimental participants would be willing to pay \$1.44 for each
additional gram of weight, and that for those who are parents, this
WTP rises to \$1.72.  Moreover, we observe a hump-shaped relationship
in WTP, with the marginal WTP becoming negative
at the top end of the range.  Interestingly, this pattern mirrors
the estimated effect of birth weight on health \citep{Caseetal2005}
and on salary \citep{BehrmanRosenzweig2004}.

Despite the positive and economically significant estimates for the
WTP for birth weight, using a range of recent results which estimate
the long-run returns to birth weight\footnote{These include
  \citet{JohnsonSchoeni2011,CookFletcher2015,BehrmanRosenzweig2004} (USA);
  \citet{Blacketal2007} (Norway); \citet{Bharadwajetal2015} (Sweden);
  \citet{RosenzweigZhang2013} (China); and \citet{CurrieHyson1999,
    Caseetal2005} Great Britain.  We provide a full discussion of these
  estimates and their methodologies in section \ref{returnsBW}.},
our back-of-the-envelope calculations suggest that the present value of the
labor market returns exceed individual, and even parental, WTP by a factor
of approximately 4 or 5. This is a puzzle, given that parents have been
shown to generally invest more in a child's health than their own
health \citep{AgeeCrocker2008}. In addition, the labor market returns
to birth weight are only a subset of the full returns: Convincing
evidence suggests that increases in birth weight reduce the prevalence
of chronic morbidities \citep{Barker1995,AlmondMazumder2005,
  JohnsonSchoeni2011b}, mortality \citep{vandenBergetal2006}, and a
range of psychological outcomes \citep{Fletcher2011}.\footnote{Here we
  focus largely on the long-run impacts of birth weight over an
  individuals' life.  However, there is an additional even larger body of
  work documenting the importance of birth weight in explaining early life
  outcomes and human capital outcomes in childhood. These include
  \citet{Almondetal2005,Oreopoulosetal2008,Guptaetal2013,Conleyetal2003,
    Figlioetal2014,LinLiu2009,Fletcher2011,Bharadwajetal2017,
    TorcheEchevarria2011} among many others.}
One potential explanation for our finding is that individuals, and even parents, under-estimate
the returns to birth weight. In an influential paper, \citet{Jensen2010}
demonstrates that correcting expectations regarding the returns to
education increases investments in education. Our results potentially
open the door to a similar phenomenon in early life health.  If parents
are not aware of the magnitude of the returns to birth weight, they may be
investing sub-optimally in pre-natal behaviors.

In what follows, we describe the Mechanical Turk data and experimental
set-up in section \ref{scn:data} and describe the methodology for
estimating parental WTP and its confidence intervals in section
\ref{scn:methods}.  In section \ref{scn:results} we discuss our
experimental estimates for the value of birth weight, and then compare
them to a series of papers which estimate the value of birth weight on
the labour market.  We briefly conclude in section \label{scn:conclusion}.

\section{Data Description}
\label{scn:data}
We collected data on preferences over birth characteristics by running
discrete choice experiments on Amazon's Mechanical Turk (MTurk) online platform.
This platform is a market place which provides access to a pool of US MTurk
workers (survey respondents) who are paid per completed ``HIT'' (Human
Intelligence Task).  We recruited a sample of respondents to complete a
discrete choice experiment (described further below) as well as a series of
demographic questions.  These demographic questions were asked after the
completion of the survey to avoid any framing or experimenter demand effects \textbf{(reference here)}.
Mechanical Turk respondents have been documented to have a series of desirable
characteristics, and be more representative of the US population than other
frequently-used subject pools such as college student samples
\citep{Berinskyetal2012}.  Mechanical Turk samples are increasingly used both
within and outside of the economics literature
\citep{Kuziemkoetal2015,Jordanetal2016}.

We recruited a sample of 1,002 respondents aged 18 or over, and conducted
the experiment on Monday September 19\textsuperscript{th}, 2016.  Workers
were paid \$1.10 for a 6 minute experimental survey (average length), resulting in
an effective hourly pay rate of approximately \$11. The survey had to be completed to be
able to receive payment, and it was impossible to move forward if the question on the screen was not answered.
We required that respondents must be from the United States,\footnote{Workers on Mechanical
  Turk are required to have a United States Social Security Number.} and in order to
maximize the likelihood that workers were based in the US at the time of
completing the survey, this was launched at 9:00 AM EST.  By 2:13 PM
of the same day 1,002 valid responses were collected.  We also required
that workers had completed at least 100 tasks on MTurk in the past, and
had achieved an approval rating of greater than 95\% on these tasks.
These restrictions are common in Mechanical Turk research
\citep{Berinskyetal2012,Francis-TanMialon2015}.  Of the 1,002 completed
respondents, we removed a small number based on a series of pre-defined
consistency checks.  These were: (a) workers whose geographical IP address
placed them outside of the US at the time of survey (36 respondents, or
3.6\%), any respondents who failed a consistency check where a question was
repeated at the beginning and end of the demographic portion of the survey
(8 respondents), and any respondents completing the entire exercise in
under 2 minutes (6 respondents).\footnote{In the appendix to this paper we
  demonstrate that nonetheless, our results remain largely unchanged if we
  do not remove these respondents.}  The final sample thus consists of 952
respondents.

The geographic location of these respondents within the United States (based
on their IP address) is provided in Figure \ref{geography}.  The geographic
coverage is broadly representative of the US population.  In appendix Table
\ref{tab:cover} we compare our MTurk respondent coverage with the US population
from 2015 \citep{CensusBureau2015}.  In general, we see that the MTurk sample
lines up well with the national population at the state level, however there
are a number of exceptions, such as the lower number of respondents from
California \textbf{(reflecting the earlier time zone difference of the West coast)}.  %In our principal specifications we always use the unweighted
%MTurk sample, however as an alternative test we also re-weight the sample
%to balance state and demographic characteristics, as suggested by
%\citet{Francis-TanMialon2015}, and discussed further in the following section. LET'S KEEP THIS FOR A RR OR NEXT SUBMISSION OF THE PAPER.

Finally, summary statistics of the respondents are provided in Table
\ref{sumstats}.  Slightly more than half of respondents are female (56\%),
and the ages of respondents range from 18 to 75 years (with a mean age of 36
years); 84\% are white and 6\% are Hispanic. Approximately half of respondents are parents (51\%). 45\% of the
respondents are married, 68\% of them are employed and 89\% have at least some college.
%\textcolor{red}{DISCUSS
%  HERE COMPARISON WITH USA POPULATION.  SHOULD WE USE ACS TO COMPARE? Alternative
%  is to just use census bureau with averages already online (from ACS) to re-weight
%  by gender, race, age and marital status(?)}
%\subsection{Birth Certificate Data}
%\subsection{Mechanical Turk Survey Data}


\section{Methodology}
\label{scn:methods}
%\subsection{Observable Determinants of Birth Weight}
%\subsection{Eliciting Willingness to Pay in a Discrete Choice Experiment}
In order to estimate the perceived importance of birth weight in terms of willingness
to pay, we run discrete choice experiments on a
large sample of US-based respondents. A discrete choice experiment (DCE) is a type of Conjoint Analysis (CA):
an experiment in which respondents are asked to choose their preferred option
from a set when a number of attributes are varied simultaneously.
CA was born from early work in consumer theory in which tastes for goods owe
to the collection of their characteristics \citep{Lancaster1966}.  In the
past, CA has been used to measure preferences over medical care in a variety of
contexts, including the valuation of waiting times \citep{Propper1990,Propper1995},
alternative miscarriage treatment options \citep{RyanHughes1997}, asthma
medications \citep{Kingetal2007}, or depression management \citep{Wittinketal2010}.
Conjoint analysis has recently been shown to be the best-performing experimental
design when compared with actual choices made \citep{Hainmuelleretal2015}.


\textbf{[CAN YOU COPY FROM SOB PAPER? For example, we should mention this is a main-effects design, etc.]}
Our birth choice experiments consist of asking respondents to consider a series
paired birth outcomes, while focusing on four attributes of each birth.  These
attributes are the sex of the child (Boy or Girl), the out of pocket expenses
associated with the birth (\$250, \$750, \$1000, \$2000, \$3000, \$4000, \$5000,
\$6000, \$7500, or \$10000), the baby's weight at birth (5lbs, 8oz; 5lbs, 13oz;
6lbs, 3oz; 6lbs, 8oz; 6lbs, 13oz; 7lbs, 3oz; 7lbs, 8oz; 7lbs, 13oz; 8lbs, 3oz;
8lbs, 8oz; or 8lbs, 13oz), and the season in which the baby is born (Winter,
Spring, Summer or Fall).  These attributes are all orthogonally varied, implying
that the effect of each characteristic on the likelihood that a particular birth
is chosen is separately identified \citep{Marshalletal2010}. Each respondent
was asked to consider 7 pairs of birth outcomes in an iterative fashion.  In
order to move forward in the experiment a choice must be made for each pair,
and once the choice has been made the respondent may not go back and revise their
choice.  In each case the two pairs
were displayed side-by-side on a single screen, and respondents were asked to
indicate which was their preferred outcome.  As well as randomising the level
of each attribute on each profile, the order of the attributes was randomised,
however to reduce the cognitive load to respondents the ordering of attributes
was only randomised once, and then fixed across the seven pairings that the
respondent ranked.  The
DCE's framing and the explanation of the attributes shown to respondents is
displayed in appendix Figures \ref{DCE-frame} and \ref{DCE-options}.  In
appendix Figure \ref{DCE-example} we display a single choice scenario as
presented to respondents.  Appendix \ref{app:procedure} describes the
survey procedure followed by respondents in more detail.

The levels of attributes were chosen to represent plausible values from the
US population \citep{RyanFarrar2000}, and extreme values were avoided to
prevent the likelihood of ``grounding effects'' (or corner solutions)
\textbf{(reference here)}.  In
order to minimise the likelihood that respondents would employ simple
heuristics in answers, we limited the number of attributes (four) which need be
considered.  Birth weights were always presented in pounds and ounces, given
that this experiment was entirely run with a US-sample familiar with the
imperial measurement system.  As well as indicating
that all births were complication-free, only birth weights over the normal
range of 2,500--4,000 grams were included (11 evenly-spaced weights were
defined in this range, and expressed in pounds and ounces in the experiment).%
\footnote{This range includes the majority of all births in the US.  According
  to full vital statistics from
  2013 (from the National Vital Statistics System), 15.91\% of births occurred
  with weights outside of this range (refer to appendix Figure \ref{bwt-nvss}.)
  Of these, 8.02\% were low birth weight ($<$ 2,500 grams), and 7.89\% were
  large for gestational age at birth ($>$ 4,000 grams).
}
An opt-out option was not included in any of the discrete choices.  This
has been suggested to have desired properties such as avoiding non-random
opt-out of all questions \citep{Veldwijketal2014,BekkerGrobetal2012}.

Consider a sample of $i\in \{1,\ldots,N\}$ individuals, each of whom consider
$K$ choice tasks in which they must decide between $J$ options (or profiles).
Each profile contains $L$ attributes, where each particular attribute $l$
consists of discrete levels of the variable.  In the case of the DCE
described above, we have $N=952$ respondents, $K=7$ choice tasks per
respondent, $J=2$ profiles per task, and $L=4$ attributes.\footnote{These
  four attributes have 2, 10, 11 and 4 levels respectively for sex, out of pocket
  costs, birth weight and season of birth.} We follow
\citet{Hainmuelleretal2013} in defining a treatment vector $T_{ijk}$.
This treatment vector has $L$ cells, and summarizes for individual $i$,
at choice task $k$, for profile $j$, the full set of attributes observed.
Each particular attribute $T_{ijkl}$ is randomly assigned from among all
the levels of $l$, the assignment of which is orthogonal to all other
attributes the respondent sees.  Using the potential outcomes framework,
we define a binary variable $Y_{ijk}(\bar{\mathbf{t}})$ which takes the
value 1 if respondent $i$ would choose profile $j$ on choice set $k$ if
faced with the set of attributes $\bar{\mathbf{t}}$, or 0 if the profile
would not be chosen.

We are interested in estimating two quantities.  Firstly, we would like to
estimate, \emph{ceteris paribus} the likelihood that a birth is chosen given
that a particular birth weight is observed (compared with an omitted base
category).  Secondly, we would like to estimate the willingness to pay for
season of birth, by combining the information from both variations in birth
weight and variations in out-of-pocket costs.

\citet{Hainmuelleretal2013} call this first quantity the Average Marginal
Component Effect (AMCE) and demonstrate that under reasonably weak
assumptions\footnote{These assumptions relate to randomization of attributes,
  and stability of respondent behaviour regardless of the number of profiles
  that they have seen or the order of the attribute in the profile.  This
  first assumption holds by construction in our experiment.  In the following
  section we return to explicitly test for violations of the latter two
  assumptions.  A benefit of the set-up of the DCE is that even if order and
  round effects are not completely neutral, these can be flexibly captured
  using fixed effects in a regression.}, it can be recovered using a
non-parametric subclassification
estimator, conditional regression, or a simple difference of means.  The
logic of the AMCE\footnote{Formally, the AMCE is defined as
  \citep{Hainmuelleretal2013}:
  \[
  E[Y_i(t_1,T_{ijk[-l]},\mathbf{T}_{i[-j]k})-Y_i(t_0,T_{ijk[-l]},\mathbf{T}_{i[-j]k})|(T_{ijk[-l]},\mathbf{T}_{i[-j]k})\in\tilde{\mathcal{T}}]
  \]
  which can be quite easily calculated by integrating over all of the other
  attributes and levels except for $t_1$ (the treatment of interest) and $t_0$
  (the baseline level for the attribute). These other attributes and levels are
  denoted as the set $\tilde{\mathcal{T}}$ here.} is to capture the change in
the likelihood that a given profile would be chosen if the
$l$\textsuperscript{th} component were changed from $t_0$ to $t_1$, or in our
case, a change in birth weight.

Under the controlled randomization in conjoint analysis, \citet{Holland1986}'s
fundamental problem of causal inference is resolved by construction, as on
average there will be no correlation between observing the particular level
of an attribute and individual correlates. Treatment units are thus those who
observe a particular $t_1$, while those who do not act as controls.  In
practice, to estimate the change in the likelihood that a birth scenario is chosen given a change in birth weight
(or any other attribute), we estimate
the following regression:
\begin{equation}
  \label{ACMEreg}
Pr(Y_{ijk}=1) = \alpha + \beta Costs_{ijk} + \sum_{l=2}^{11} \gamma_l BW_{ijk,l} + \sum_{l=2}^{4} \delta_l SOB_{ijk,l} +  \kappa Girl_{ijk} + \mu_j + \phi_k + \varepsilon_{ijk}.
\end{equation}

Here we can include option and profile order fixed effects ($\mu_j$ and
$\phi_k$ respectively), %\footnote{In practice, if there are no round or position effects, rather than estimating the conditional regression in equation \ref{ACMEreg}, we could simply take the difference of means. However
%given that we find weak evidence that there is some instability in choices
%between rounds (appendix figures \ref{DCE-asm2}-\ref{DCE-asm3}), we elect to
%include controls for $\mu_j$ and $\phi_k$.},
and standard errors are clustered at the level of the respondent to capture the (likely) positive correlations
among choices based on attributes by a particular respondent. We estimate
equation \ref{ACMEreg} using a logit model and report average marginal
effects, though note that findings are not qualitatively different if
estimated as a linear probability model.
%\footnote{This
  %will capture, for example, that a respondent who is encouraged to choose
  %a particular birth given a higher birth weight being similarly likely to
  %choose other birth options when observing higher birth weights.}
The coefficients $\gamma_l \ \forall\ l$ are the principal effects of interest,
and capture the likelihood that a birth is chosen given a particular birth
weight.  We omit from the regression the lowest birth weight category as the
baseline level, implying that all coefficients should interpreted as the
marginal likelihood of choosing a birth given birth weight $l$ in place of
the lowest birth weight (2,500 grams).

In the above model, we are also able to estimate the willingness to pay for birth weight.  Consider the two AMCEs, $\beta$ and $\gamma_2$
(where $\gamma_2$ is chosen without loss of generality).  These coefficients
represent the marginal effect on the likelihood of choosing a particular birth
given an increase in the particular attribute, conditional on all other
attributes:
\[
\beta=\frac{\partial Pr(Y_{ijk}=1)}{\partial Costs_{ijk}} \qquad \gamma_2=\frac{\partial Pr(Y_{ijk}=1)}{\partial BW_{ijk,2}}.
\]
The marginal rate of substitution between the particular birth weight $BW_2$
and the price of a given birth (the out of pocket costs) thus gives the change
in costs that an average respondent would be willing to withstand for a marginal
increase in birth weight:
\[
MRS_{BW_2,Costs}=\frac{\frac{\partial Pr(Y_{ijk}=1)}{\partial BW_{ijk,2}}}{\frac{\partial Pr(Y_{ijk}=1)}{\partial Costs_{ijk}}}.
\]
The quantity is precisely the willingness to pay, that is, the change in financial
resources that makes a respondent indifferent between the higher or lower birth
weight:
\[
WTP_{BW_2}=-\frac{\gamma_2}{\beta}=-\frac{\partial Cost_{ijk}}{\partial BW_{ijk,2}}.
\]
Note that in the above calculation we take the negative so that costs are
interpreted as the (positive) amount that must be paid rather than the
(negative) change in financial resources.  This $WTP_{BW}$ can also be
derived quite straightforwardly from a model of the indirect utility
function as described in \citet{Zweifeletal2009}, and applied in
\citet{Clarkeetal2016} to estimate the value of season of birth.

Unlike the calculation for the ACME for birth weight, the above WTP
is not associated with a standard error and confidence interval.  In
order to calculate the confidence intervals associated with the WTP
we use the delta method, which is both simple and shown to perform
well under simulation \citep{Hole2007}.  We also find that these
confidence intervals are quite comparable to those produced when using
block bootstrapping, as outlined further in the following section.

\section{Results}
\label{scn:results}
%\subsection{The Experimental Valuation of Birth Weight}
%\label{sscn:expR}
\subsection{Principal Experimental Results}
\subsubsection{Average Marginal Component Effects}
In Figure \ref{DCE-samp} we present our experimental results.  This figure
displays point estimates of the likelihood of preferring a particular
birth scenario given each characteristic, compared with an omitted base category
for each characteristic.  Along with each point estimate, the 95\%
confidence interval is plotted, clustering by respondent.  While we present cost as a linear variable measured in 1,000s of
dollars, in appendix Figure \ref{DCE-full-samp} the same results are
presented with costs displayed as the same categorical measure observed
by respondents.

The top panel displays the likelihood of choosing a birth given a
particular birth weight, compared to being shown the minimum sample
birth weight of 5lbs, 8oz (2,500 grams).  In each case, higher birth
weights are associated with a greater likelihood of choosing the
birth.  The most preferred birth weight (based on point estimates) is
7lbs, 8oz (3,400 grams), which results in a birth being approximately
18pp more likely to be chosen than the omitted base category.  The
magnitudes of the estimates are large.  With the exception of 5lbs, 13
oz, all higher birth weights are at least 10\% more likely to be chosen,
and in each case the difference is statistically significant.  In addition, there appears to be a hump-shaped pattern, with the most
preferred births being those towards the middle of the (normal) birth
weight range, and lower preferences for those at the extremes of the
normal weight range.  %We return to this point below when discussing the estimates for the willingness to pay for birth weight.

\subsubsection{Willingness to Pay for Birth Weight}
As discussed in section \ref{scn:methods}, we can combine estimates of
characteristics with those on out of pocket costs to generate estimates
of the willingness to pay for each characteristic.  In Table \ref{WTPreg}, column 1,
we assume a linear functional form for birth weight.  By comparing the change
in the likelihood of choosing a birth based on an increase in birth weight
with the change in likelihood due an increase in costs, we estimate that
the willingness to pay for an additional 1,000 grams in the full sample
is \$1,438.3, or \$1.44 per gram.  However, as we observe in column 2,
the relationship between birth weight and likelihood of choosing a birth
is non-linear.  In Figure \ref{WTP-relative} we document the WTP of all
birth weight options, with respect to the minimum birth weight in the
sample.  We observe that the largest relative difference occurs at 3,400
grams (compared with the omitted base of 2,500 grams), with a WTP of
nearly \$3,000USD, or \$2.14 per gram. Finally, and as expected, we observe that all else equal, higher costs result in
a birth being less likely to be preferred.  On average, for each additional \$1,000 in out of pocket expenses, the likelihood of choosing a birth falls by
nearly 10\%.  The non-linear estimates of these parameters are displayed in appendix figure \ref{DCE-full-samp}.


%Similarly, in table \ref{WTPgreg} we present estimates splitting by the gender of the respondent.  Interestingly, the WTP for
%birth weight is actually greater among male (\$1.66 per gram) rather than
%female (\$1.28 per gram) respondents.  While female respondents are much
%more likely to choose a birth when falling in a favourable season, male
%respondents are more likely to choose a birth given greater birth weight.
%While not conclusive, it is worth noting that these results are in line with
%the fact that the impacts of a baby's birth weight affect both mothers and
%fathers in a similar way, while the costs of birth season (and hence timing
%of pregnancy) fall much more on mothers who experience pregnancy in
%different seasons of the year, perhaps rationalising the observed
%differences in columns 4 and 5.

\paragraph{Non-parametric WTP estimates.} When calculating the WTP of birth weight as a single figure, this is
based on a specification in which birth weight (and costs) enter the
estimating equation linearly.  However, in Figure \ref{WTP-relative} we
display the fully non-parametric estimates where birth weight enters
the model in the same categories as displayed to experimental participants.
Figure \ref{WTP-relative} provides the WTP for each particular birth
weight, as compared to a birth weight of 2,500 grams (or the relative
willingness to pay).  In Figure \ref{WTP-marginal} we also present the
marginal WTP, or the difference in WTP when moving between contiguous
categories.  It is observed that the largest marginal change occurs
at the lower end of the distribution, when moving from approximately
2,600 to 2,800 grams, and that at the upper end of the birth weight
range, the marginal WTP turns (significantly) negative, when moving
from 3,800 to 4,000 grams.  Interestingly, this hump-shaped relationship
lines up with a range of evidence related to the \emph{benefits} of
birth weight.  For example, \citet{Caseetal2005} show a similar
hump-shaped relationship for the effect of birth weight on health,
with increases in (self-reported) adult health as birth weight
increases, though a point of inflexion exists at very high birth weights
(see for example their figure 1).\footnote{Indeed, this hump-shaped pattern is commonly observed in many morbidity and
  mortality measures across populations \citep{Wilcox2001}.} Similarly,
\citet{BehrmanRosenzweig2004}
find a point of inflection of the labor market returns to birth weight
which lines up quite closely with our estimates (specifically, they
find that for births approximately 2 standard deviations above the mean
of fetal growth\footnote{Although these values can not be precisely
  converted to birth weight from the rate of fetal growth reported by
  \citet{BehrmanRosenzweig2004}, an approximate calculation suggests
  that if fetal growth is a good proxy for birth weight and hence the
  turning point for birth weight is similarly two standard deviations
  above the mean, this would result in a turning point of
  $90.2 oz + 2\times 17.9 oz = 126 oz$, or 3.572 kg (all values from
  \citet{BehrmanRosenzweig2004} Table 1 and Figure 8).}, labor market
returns to a marginal increase in birth weight turn from positive to
negative). We return to these estimates further below in section
\ref{returnsBW}.

\paragraph{Alternative confidence intervals for WTP estimates.} 
The confidence intervals estimated on willingness to pay discussed
above are always calculated using the delta method.  However, we find
relatively little difference if we estimate confidence intervals under
block bootstrapping.  In order to implement this block bootstrap
procedure we perform 1,000 bootstrap replications, resampling with
replacement over experimental \emph{respondents} rather than over
profiles.  This block bootstrap calculation leads to a 95\% confidence
interval for willingness to pay of [\$1,107.2;\$1,769.3] for the main
sample.  This is marginally wider, though qualitatively very similar to
the main calculation using the delta method reported in Table \ref{WTPreg}
of [\$1,119.4;\$1,757.1].  In all cases examined, block bootstrap confidence
intervals lead to largely similar findings, and in some cases even lead to
slightly \emph{less} wide confidence intervals. Full comparisons of
confidence intervals between methods are available in appendix Table
 \ref{bootstrapCI}.

%SHOULD WE MENTION THE ``13 EFFECT''?
%\paragraph{Population Re-weighting}

 \paragraph{The value of other attributes.}
 Table  \ref{WTPreg} also sheds light on preferences for other birth
 attributes. We find no evidence
of any elicited preference for the baby's gender on average.
Indeed, in both specifications displayed in table \ref{WTPreg} estimated coefficients on the baby being a girl are
quite tightly estimated zeros (ranging from 0.000 to 0.001 with clustered
standard errors of 0.010). However, when estimating separately by the gender of the respondent, we do observe a preference for
boy children among male respondents (table ??? \textbf{APPENDIX}). This is in agreement with the results of \citet{DahlMoretti2008} who document a demand
for sons, particularly among fathers (table ??? \textbf{NEW TABLE FOR FATHERS IN THE APPENDIX}).  When considering season of birth we observe a greater likelihood to choose births in the spring, evidence
of a demand for certain seasons of birth \citep{Clarkeetal2016}.


\subsection{Private WTP, Public WTP, and ``The Returns to Birth Weight''}
\label{returnsBW}

\subsubsection{Comparison with the Public WTP Estimated from a Targeted Program}
Our main findings suggest that individuals are willing to pay \$1.44 for each additional gram of birth weight over the normal birth weight range, and this increases to \$1.72 among parents.
It is of interest to ask how this private WTP compares
with the inferred WTP from public investment.  While much of the benefits
of increases in birth weight accrue to families, such as private returns
on the labor market, a reduction in out of pocket medical spending during
childhood, and increases in education \citep{BehrmanRosenzweig2004,
  Oreopoulosetal2008}, increases in birth weight also have
important public returns: benefits flowing from reductions in public health care spending,
and lower usage of means-tested public benefits programs
\citep{Almondetal2005,Bharadwajetal2015}.

We can provide a rough comparison between public and private WTP for
birth rate using estimates from the Special Supplemental Nutrition
Program for Women, Infants, and Children (WIC), which provides food and education to pregnant and postpartum
breastfeeding women who earn less than 185\% of the US federal
poverty guideline.\footnote{ A comprehensive discussion of recent work on
WIC is available in \citet{BitlerKaroly2015}.}  By combining estimates
of the cost per WIC user with estimates of the benefit in terms of
additional birth weight, we can arrive at a rough estimate of the
public WTP per gram of birth weight.

\citet{BenShalometal2011} document that WIC participation costs
\$54 per enrollee per month, and according to WIC administrative
data, 56.9\%, 34.7\% and 7.8\% of participants enroll in the first,
second or third trimester respectively \citep{Johnsonetal2013}.  Using trimester midpoints
to calculate months of enrollment, this suggests total approximate
costs of covering a single pregnant women of \$321.\footnote{This
  is calculated
  as \[
  Cost = 54\times (7.5\times0.569+4.5\times0.347+1.5\times0.078) = \$321.1.
  \].}
There exist a very large range of estimates of the impact of the WIC
program---with some estimates suggesting effects of up to 185 grams
\citep{KowaleskiJonesDuncan2011}---however more recent studies suggest
that the true impact may fall towards the lower end of the estimated
spectrum.  Among plausibly causal estimates, \citet{RossinSlater2013}
estimates that participation has a mean impact of 27 grams of birth weight,
and \citet{Hoynesetal2011} estimate impacts of 18-29 grams. In the case of
the highest estimated impact, public
WTP equates to \$321/185 grams = \$1.74 per gram, while at the lower
end of the spectrum, the WTP equates to \$321/18 grams = \$17.83 per gram. Regardless of the estimates used, the public WTP exceeds our experimental estimates of the private WTP, for both the whole sample of respondents (\$1.44), and the subsample of parents (\$1.72).

%From Royer:
%``The fetal origins hypothesis has many important economic implications. It would
%suggest that policies that aim to improve the prenatal environment (e.g., Medicaid
%expansion for pregnant women and the Special Supplemental Nutrition Program for
%Women, Infants, and Children (WIC)) could reap large long-run health, human capital,
%and wage returns for both current and future generations. Such effects are often
%ignored in cost-benefit calculations, potentially leading to an underinvestment in
%these programs.''

\subsubsection{Comparison with the Public WTP Estimated from an Untargeted Program}
The evidence from WIC discussed above estimates the inferred WTP using a
targeted program which explicitly focuses on maternal and newborn health.
Nevertheless, there are a range of other public programs which, while not
explicitly targeting infant health, have been documented to have (unintended)
effects on these outcomes.  Perhaps the largest of these was the Food Stamp
Program (or FSP), now known as the Supplementary Nutrition Assistance Program
(SNAP).\footnote{A number of large means-tested public programs flowing from
  the ``War on Poverty'' starting in 1964 have been documented to have a
  diverse range of direct and indirect impacts (as outlined in the previously
  mentioned \cite{BitlerKaroly2015} review.  A comprehensive review of the
  FSP and other programs is also available in \citet{BaileyDanziger2013}.}
For simplicity, in what follows we will always refer to the program as The FSP.
The FSP is one of the largest social safety net programs in the US, providing
support for 44.2 million people in 2016 at a total cost of 70.9 billion dollars.

\citet{Almondetal2011} provide a particularly well-identified estimate of the
effect of the FSP on infant health, and in particular, on birth weight.  Using
county-level variation in program roll-out\footnote{Of note, this identification
  strategy is broadly similar to that used by \citet{Hoynesetal2011} in estimating
  the impact of the WIC program, ensuring that estimates in each of these sections
  of this paper are comparable.} and vital statistics data from 1968-1977 (a
period of sharp program expansion), they estimate that the average county-level
birth weight in program counties increased by between 2-2.6 grams for white
pregnant women and 1.7 to 5.5 for black pregnant women.  While this estimate is
a county-level valye, they also provide an \emph{individual}-level effect based
on county usage rates.  These individual effects of 20.27 grams (white) or 31.69
grams (black)\footnote{Refer to column 2 of table 1 from \citet{Almondetal2011}.}
allow us to estimate the inferred public WTP for a gram of birth weight when
combined with the costs per pregnant women.

In order to determine the costs per pregnant women, we focus on data on
current costs and users (in order to be comparable to our estimated WTP
in current dollars).  We assign the full cost per program beneficiary for
three months of use.  While this is a rough estimate (and we note below
the WTP under alternative assumptions, including 9 months of coverage), we
do use this value for a number of reasons. Firstly, there is considerable
evidence that nutrition can affect birth weight (rather than survival or
other morbidities) only over a relatively short window, in particular
during the third trimester (see for example
\citet{StephensonSymonds2002}, as well as evidence from the timing of
exposure to the Dutch Famine \citep{Schulz2004}).  In particular, this
is also shown to be the case with the FSP \citep{Almondetal2011}.
Secondly as the FSP is not targeted to pregnant women or to promote
neonatal health \emph{per se}, the total cost should not be....



\subsubsection{Returns to Birth Weight on the Labour Market}
Although there exists little or no evidence on the parental valuation of
birth weight, a broad literature from various fields has documented
the returns to birth weight.  It is well accepted that higher birth
weight is associated with reductions in morbidity and mortality, and
greater educational attainment and achievement throughout
childhood.\footnote{For example, on morbidity, refer to
  \citet{Almondetal2005,Oreopoulosetal2008,Guptaetal2013,Conleyetal2003},
  and on early-life education, among others \citet{Figlioetal2014,
    LinLiu2009,Fletcher2011,Bharadwajetal2017,TorcheEchevarria2011}
  demonstrate a strong and plausibly causal link.}  Moreover, these
impacts have been well-documented to perdure into adulthood and impact
labor market returns.  %Recent large-scale evidence from Sweden suggests
%that in some contexts at least, labour market returns to birth weight do
%not fade out until at least 50 years of age or greater
%\citep{Bharadwajetal2015}.

In Table \ref{litrev} we review the range of papers which have estimated
the long-run returns to birth weight.  The data requirements for such an
exercise are quite demanding, requiring information on the weights of
babies at birth, family linkages, and completed education or labor market
outcomes many years later.  Nevertheless, in a range of contexts in USA,
China, Norway, Canada and the UK this has been possible.  Of
particular interest for this paper are the returns to birth weight on
the labor market in the US.  One way to benchmark the parental WTP for child
birth weight is to determine how it compares to the present value of the
flow of expected benefits during the life of their child. Thus,
considering these well-estimated cases of the labor market returns to
birth weight, we can discount expected returns back to the start of an
individual's life, and compare it with our experimentally estimated
WTP.\footnote{This should of course be considered as a lower bound to
  the true value of birth weight.  Labor market returns are a convenient
  financial metric, but do not include any of the additional pecuniary or
  non-pecuniary benefits which may flow to parents from a higher birth weight
  child such as lower expected costs associated with medical care
  \citep{Almondetal2005}.}


%This exercise is well suited to the experiment documented in section \ref{sscn:expR} of this paper.
The most convincing empirical estimates
produced from the literature come from within-sibling or within-twin
methods, which can be viewed as the effect of shifting the smaller of
two siblings (or twins) to the weight of the larger sibling (twin).
Our experimental estimates of WTP are taken as the relative to the
omitted baseline category of 2,500 grams.  So, in the sense that we
think of 2,500 grams as the lower weight of a comparison pair, all
our experimental estimates line up to the effect of additional
intrauterine growth reported by the within twin or within family
literature reviewed in Table \ref{litrev}.

For this exercise, we are most interested in those papers which
provide estimates of the long-run returns to birth weight on the
labor market.  This precludes studies which only observe completed
education, but not labor market outcomes
described in panel B of table \ref{litrev}. %\footnote{A notable
 % exception is the study of \citet{Royer2009}, which although only
 % having individual level on education, also has income at the zip
 % code of residence of the individual.  As we discuss further below,
 % we use the zip code averages of income as a consistency check for
 % our methodology.}
 Among those papers which have estimated the effect
of birth weight on labor market returns, there are five papers that
use twin or sibling fixed effects to leverage within family variation
in birth weight to estimate returns conditional on genetic material.
These are \citet{BehrmanRosenzweig2004,JohnsonSchoeni2011,CookFletcher2015},
using data from the USA, \citet{Blacketal2007} using Norwegian data, and
\citet{RosenzweigZhang2013} with Chinese data.  Of these 5, all with
the exception of \citet{JohnsonSchoeni2011} estimate
the impact of increases in (continuous) birth weight rather than
a binary indicator for low birth weight (LBW), or birth weights
inferior to 2,500 grams.  Of those four studies which use a continuous
measure of birth weight and between twin estimation strategy, all find
that headline effects of birth weight on labor market returns are positive
and significant.

In order to generate a back-of-the-envelope comparison of the WTP
for birth weight with the present value of expected labor market
returns, we begin with the estimates of \citet{BehrmanRosenzweig2004}
from the USA.  \citet{BehrmanRosenzweig2004}'s results provide
a point estimate of the labor market returns to birth weight which
suggests that ``augmenting a child's birth weight by a 1 lb.\ increases
her adult earnings by over 7\%''.  According the \citet{USCB2016},
the median personal income in the US in 2015 was \$30,240.  If we assume
a working life which begins at the age of 25 and ends at the age of
60, we can calculate the present value of a 7\% increase in wages as
a deferred annuity.  This calculation suggests that, based on the
estimates of \citet{BehrmanRosenzweig2004}, the present value of an
additional pound of birth weight is \$10,235.\footnote{We calculate the
  present value as
  \[
  PVBW = (\$30240\times0.07)\times\frac{1-(1+0.05)^{-35}}{0.05}\times\frac{1}{(1+0.05)^{25}}=\$10235.46
  \]
}  Dividing this value by the 454 grams in a pound gives the labor
market value of a gram of weight of $\$23$.  If we assume
that only approximately 60\% of the working age population will
actually be employed in the labor market \citep{BLS2017}, scaling
by this value still suggests a labor market return of approximately
\$14, an order of magnitude higher than our estimated values of WTP.

This back-of-the-envelope calculation using
\citet{BehrmanRosenzweig2004}'s estimates is a very crude estimate,
relying on a number of assumptions that are unlikely to hold in
practice.  Chief among these is that the returns to birth weight
are stable over the life course, and salary and labor market participation
rates are also stable over the life course. Still, they are informative if only because the \$14 per gram is close to the public WTP of \$17.83 per gram,
albeit is 8-10 larger time than the private WTP estimated among our respondents.


\subsection{Explaining the Divergence between Private WTP, Public WTP and Returns to Birth Weight}
In this section we present a very stylized framework based on Becker and Tomes (1979) and Solon (2004) to understand our estimates of the WTP for birth weight, and its potential policy implications.  

\subsubsection{WTP for Birth Weight}

We focus on families with one parent and one child, disregarding bargaining issues in the family (see Browning, Chiappori and Weiss, 2014). In each family $i$, parents must decide how much to consume $C_{i,t-1}$ and invest $I_{i,t-1}$ in a specific form of child human capital, namely, birth weight ($BW_{i,t}$). Parent's lifetime earnings are $Y_{i,t-1}$, and the parental budget constraint is given by
\begin{align*}
Y_{i,t-1} = C_{i,t-1} + I_{i,t-1}
\end{align*}
In other words, parents cannot borrow against the child's prospective earnings and do not leave bequests (financial assets) to the child; parents are credit-constrained.\footnote{This assumption can be relaxed. See Becker and Tomes (1986).}

The child income generating function is given by
\begin{align*}
\log(Y_{i,t}) = \alpha + \beta \log(BW_{i,t}) + u_{i,t}
\end{align*}%\pause
where $\beta$ is the return to birth weight in the labor market and $u_{i,t}$ is a random shock to earnings.%\pause

If parents preferences can be represented by a Cobb-Douglas utility function 
\begin{align*}
U_i = \gamma \log(C_{i,t-1}) + (1-\gamma) \log(Y_{i,t})
\end{align*}%\pause
where $(1-\gamma) > 0$ is the altruism parameter, then the willingness to pay for $BW_{i,t}$ is given by
\begin{align}\label{WTP1}
WTP_{BW_{i,t}}=\frac{d C_{i,t-1}}{d BW_{i,t}}|_{dU_i=0}=\left(\frac{1-\gamma}{\gamma}\right)\beta \left(\frac{C_{i,t-1}}{BW_{i,t}}\right)
\end{align}%\pause

 

This simple model has two main predictions:
\begin{itemize}
  \item The higher is the return to birth weight, the higher will be the WTP for birth weight.
  \item The higher is altruism, the higher will be the WTP for birth weight.
\end{itemize}
%If one is willing to assume that both parents and non-parents have the same information regarding $\beta$, and in particular, that their estimate of $\beta$ is similarly downward biased, this model tells us that the lack of information regarding $\beta$ explains ... In addition, we found that non-parents have a lower WTP than parents. Ceteris paribus, this may well be explained by the fact that parents exhibit a higher altruism towards children than non-parents.

\subsubsection{Optimal Parental Investment in Birth Weight}
To find the optimal level of parental investment in birth weight, we need information regarding the production of birth weight. Suppose this is given by the following technology
\begin{align*}
\log(BW_{i,t}) = \theta \log(I_{i,t-1}) + e_{i,t}
\end{align*}
where $\theta >0$ denotes the efficiency/productivity of parents in translating investments into birth weight and  $e_{i,t}$ denotes the genetic endowment the child receives regardless of the investment choices of the family, so that $\log(I_{i,t-1})$ and $e_{i,t}$ are uncorrelated.%\footnote{ $e_{i,t}$ is ``good luck'' (if positive) and ``bad luck'' (if negative).}

Given the parental budget constraint, the child income generating function, and the technology to produce birth weight, the parental maximization problem can be rewritten as
\begin{align*}
\max_{ I_{i,t-1}}  U_i = \gamma \log(Y_{i,t-1} - I_{i,t-1}) + (1-\gamma) \alpha + (1-\gamma) \beta \theta  \log(I_{i,t-1}) + (1-\gamma) u_{i,t} + (1-\gamma) \beta e_{i,t}
\end{align*}
The FOC for maximizing utility
\begin{align*}
\frac{\partial U_i}{\partial I_{i,t-1}} = -\gamma\frac{1}{Y_{i,t-1} - I_{i,t-1}}  + (1 - \gamma) \beta \theta \frac{1}{I_{i,t-1}} = 0
\end{align*}%\pause
Solving for the optimal choice of $I_{i,t-1}^*$ yields
\begin{align}\label{INV}
I_{i,t-1}^* = \left[\frac{(1-\gamma) \beta \theta}{\gamma + (1-\gamma) \beta \theta} \right]Y_{i,t-1}
\end{align}

This simple model has four main predictions:
\begin{itemize}
\item The higher is the return to birth weight, the higher will be the level of parental investment in birth weight.
\item The higher is altruism, the higher will be the level of parental investment in birth weight.
\item The higher is parental income, the higher will be the level of parental investment in birth weight.
\item The higher is parental efficiency/productivity, the higher will be the level of parental investment in birth weight.
\end{itemize}
 
Once we know (\ref{INV}), we can find the optimal birth weight and parental consumption

\begin{align*}
\log(BW_{i,t}^*) = \theta \log \left(\frac{(1-\gamma) \beta \theta }{\gamma + (1-\gamma) \beta \theta} Y_{i,t-1} \right) + e_{i,t}
\end{align*}
\begin{align*}
BW_{i,t}^* = \left(\frac{(1-\gamma) \beta \theta }{\gamma + (1-\gamma) \beta \theta} Y_{i,t-1} \right)^{\theta} \exp(e_{i,t})
\end{align*}
and parental consumption
\begin{align*}
C_{i,t}^* = Y_{i,t-1} - I_{i,t-1}^* = \frac{\gamma}{\gamma + (1-\gamma) \beta \theta}Y_{i,t-1}
\end{align*}

Hence, the WTP for birth weight at the optimum is given by
\begin{align}\label{WTP1}
WTP_{BW_{i,t}}^*=   \left(\frac{1-\gamma}{\gamma}\right)\beta \left(\frac{\frac{\gamma}{\gamma + (1-\gamma) \beta \theta}Y_{i,t-1}}{\left(\frac{(1-\gamma) \beta \theta }{\gamma + (1-\gamma) \beta \theta} Y_{i,t-1} \right)^{\theta} \exp(e_{i,t})}\right)
\end{align}%\pause

We have now that at the optimum the WTP for birth weight depends on 3 parameters ($\gamma$, $\beta$, $\theta$), parental income and a random shock (unobservable to the econometrician, such as parental birth weight?). 

 
\textbf{QUESTIONS?}
\begin{enumerate}
  \item Comparative statics of this expression? 
  \item In our sample we have a proxy for $Y_{i,t-1}$ (in brackets)
  \item $(1-\gamma)$ \textbf{different} between parents and non-parents?
  \item $(1-\gamma)$ \textbf{similar} between parents and non-parents who are pregnant or plan to have children?
  \item $\beta$ \textbf{similarly} perceived by individuals, regardless of their parental status: it should not be difficult to argue that both parents and non-parents have not much knowledge of the returns to birth weight in the labor market.
  \item $\theta$ \textbf{different} depending on educational level (Grossman, 1972): more educated individuals are more efficient at producing child health.\footnote{Grossman (1972): Education is assumed to improve the efficiency to produce health investments (better knowledge of harmful effects of smoking; better ability to follow medical instructions; etc.)}
  \item $e_{i,t}$ can be simulated using a normal distribution to mimic birth weight in the US population?
\end{enumerate}

\subsection{Heterogeneous Effects}
\paragraph{Parents vs. non-parents.} In Table \ref{WTPgreg} we observe that the WTP is highest among \emph{parents} (those who report having children),
at \$1.72 per gram, compared to \$1.17 among those who are not (currently)
parents (those who report having no children).  Perhaps unsurprisingly, parents are much more likely than
non-parents to be swayed by changes in non-pecuniary attributes: for parents
birth weight and birth season are considerably more important than for
non-parents. %, while non-parents place marginally greater weight on an increase in costs.



\newpage

%Talk about Royer as an additional test (looking at average income in area)

\section{Conclusion}
\label{scn:conclusion}

\newpage
\noindent\textbf{Notes}: This experiment documented in this paper has passed ethical approval at the Oxford Centre of Experimental Social Sciences (CESS), and been registered as project ETH-160128161.

\bibliography{./refs}

\clearpage
\end{spacing}
\section*{Figures and Tables}
\begin{figure}[htpb!]
  \begin{center}
    \caption{Geographic Coverage of Respondents}
    \label{geography}
  \includegraphics[scale=0.9]{../results/DCE/Summary/surveyCoverage.eps}
  \end{center}
  \floatfoot{\textsc{Notes:} The full survey sample consists of 1,002 respondents.  The final estimation sample consists of 952 respondents given that it removes respondents whose geographic IP suggested a non-US location (36 respondents, 3.6\%), those who failed to respond that their educational attainment was identical at the beginning and end of the survey (8 respondents, 0.8\%), and those who completed the discrete choice experiment in under two minutes (6 respondents, 0.6\%).}
\end{figure}

\begin{table}[htpb!]
  \begin{center}
    \caption{Summary Statistics of Experimental Respondents}
    \label{sumstats}
    \begin{tabular}{lccccc} \toprule
    \input{./../results/DCE/Summary/MTurkSum-clean.tex}
    \bottomrule
    \multicolumn{6}{p{12.2cm}}{{\footnotesize\textsc{Notes:} Refer to figure \ref{geography} for a discussion of the experimental sample. Years of education, total income and hourly MTurk earnings are calculated from categorical variables.}}
  \end{tabular}
  \end{center}
\end{table}

\clearpage

\begin{figure}[htpb!]
  \begin{center}
    \caption{Discrete Choice Experimental Results}
    \label{DCE-samp}
  \includegraphics[scale=0.9]{../results/DCE/Figures/Conjoint_Sample_continuous.eps}
  \end{center}
  \floatfoot{\textsc{Notes:} Point estimates and confidence intervals are displayed of the change in likelihood of choosing a birth profile given that a particular characteristic was seen.  Each characteristic is compared to the omitted base case indicated on the zero line.  Each respondent observes 7 paired birth scenarios, resulting in 14 profiles per respondent.  95\% confidence intervals are clustered by respondent, and costs are displayed as a linear coefficient.  Fully non-parametric costs are displayed in appendix figure \ref{DCE-full-samp}.}
\end{figure}

\input{./../results/DCE/Regressions/conjointWTP.tex}

\begin{figure}[htpb!]
  \begin{center}
    \caption{Relative Willingness to Pay for Birth Weight}
    \label{WTP-relative}
  \includegraphics[scale=0.74]{../results/DCE/Figures/WTP_relative.eps}
  \end{center}
  \floatfoot{\textsc{Notes:} Each point and confidence interval are with respect to the baseline (omitted) category of 2,500 grams, the minimum displayed birth weight.  Willingness to pay is determined as the ratio between the particular birth weight and out of pocket costs estimated as average marginal effects in a logit regression.  95\% confidence intervals displayed are calculated using the delta method.}
\end{figure}

\begin{figure}[htpb!]
  \begin{center}
    \caption{Marginal Willingness to Pay for Birth Weight}
    \label{WTP-marginal}
  \includegraphics[scale=0.74]{../results/DCE/Figures/WTP_marginal.eps}
  \end{center}
  \floatfoot{\textsc{Notes:} Each point and confidence interval compare the willingness to pay for a particular birth weight compared to the preceding birth weight.  Willingness to pay is determined as the ratio between the particular birth weight and out of pocket costs estimated as average marginal effects in a logit regression. 95\% confidence intervals displayed are calculated using the delta method.}
\end{figure}

\begin{landscape}
  \begin{longtable}{p{5.5cm}p{2.2cm}p{2cm}p{2cm}p{2.7cm}p{2cm}p{3.4cm}}
    \caption{Estimates of the Labour Market Returns to Birth Weight} \label{litrev} \\
    \toprule
    Authors&Weight&Geographic&Time  &Estimated&Denominator&Estimation \\
           &Range &Area      &Period&Return   &           &Strategy   \\
    \midrule
    \endfirsthead

    \multicolumn{7}{c}{ \tablename\ \thetable{} -- continued from previous page} \\
    \midrule
    Authors&Weight&Geographic&Time  &Estimated&Denominator&Estimation \\
           &Range &Area      &Period&Return   &           &Strategy   \\
    \midrule
    \endhead
    \midrule\multicolumn{7}{r}{{\textbf{Continued on next page}}} \\
    \multicolumn{7}{p{22cm}}{\begin{footnotesize}$^{a}$ No wage results, years of completed education used. $^{b}$ Labour market participation indicator. $^{c}$ Earnings expressed as natural logarithm. $^{d}$ Standard error is calculated based on $t$-statistic reported in original paper. $^{e}$ Binary indicator for timely graduation from high school.  Results by birth weight groups are presented with respects to $>$3,500g. \end{footnotesize} } \\ \midrule
    \endfoot
    \midrule\multicolumn{7}{p{22cm}}{\begin{footnotesize}$^{a}$ No wage results, years of completed education used. $^{b}$ Labour market participation indicator. $^{c}$ Earnings expressed as natural logarithm. $^{d}$ Standard error is calculated based on $t$-statistic reported in original paper. $^{e}$ Binary indicator for timely graduation from high school.  Results by birth weight groups are presented with respects to $>$3,500g. \end{footnotesize} } \\    \bottomrule
    \endlastfoot

    \multicolumn{7}{l}{\textbf{Panel A: Labour Market}} \\
    \citet{BehrmanRosenzweig2004} &$\mu=90.2$oz ($\mu=2,557$)&Minnesota&1936-1955&0.190(0.077)$^{c,d}$&oz/week pregnancy&Between MZ twin\\
    \citet{Blacketal2007} &$\mu=2,598$&Norway&1967-1977&0.12(0.06)&ln(Birth Weight)&Between twins\\
%    \citet{Bharadwajetal2015} &$\mu=2,641$ M $\mu=2,549$ F&Sweden&1926-1958&0.0893(0.0394)&ln(Birth Weight)&Between twins\\
    \citet{RosenzweigZhang2013}&$\mu=2,540$ M $\mu=2,430$ F&China&1973-1984&0.195(0.079)$^{d}$ M 0.054(0.104)$^{d}$ F&1,000g&Between twins\\
    \citet{CookFletcher2015} & $\mu=3,367$&Wisconsin&1957 HS graduates&0.0997(0.0788)&Birth Weight (1 sd)&Between siblings\\
    \citet{CurrieHyson1999} &Pr(LBW)= 0.045-0.088&Great Britain & March 1958 &-0.017(0.029) M -0.035(0.029) F& LBW&Observable controls\\
    \citet{Caseetal2005} &Pr(LBW) =0.073 & Great Britain & March 1958 & -0.037(0.021)$^{b}$& LBW & Observable controls \\
    \citet{JohnsonSchoeni2011} &NA&USA (PSID)&1951-1975&-0.1667(0.097)$^{c}$&LBW&Between siblings (males only) \\
    \multicolumn{7}{l}{\textbf{Panel B: Completed Education}}\\
    \citet{Royer2009}&$\mu=2,533$&California&1960-1982&0.16(0.07)$^{a}$ &1,000g (3500-2500g)&Between twins (females only)\\
    \citet{Oreopoulosetal2008} &$\mu=2,517$&Manitoba&1979-1985&0.069(0.061)$^{e}$&1,000g&Between twins\\
    \citet{CurrieMoretti2007} &$\mu=3,268$&California&1970-1974&-0.079(0.014)$^{a}$&LBW& Between siblings (females only)\\
    \citet{ConleyBennet2000} &Pr(LBW) =0.07&USA (PSID)&1968-1973&-2.024(0.764)$^{e}$&LBW&Between siblings\\
  \end{longtable}
\end{landscape}
%Royer: <2,500g: -0.02 education effect.  2,500g +: 0.33 effect.
%NOTE: Royer 2009 and Currie Moretti: could use income of zip code.
%NOTE; Johnson Schoeni estimate $4819 per year with Tobit model.
%reaching grade 12 by 17 years

\input{./../results/DCE/Regressions/conjoint-gendInd.tex}

%\begin{table}
%  \caption{Back-of-the-Envelope Calculation of the Discounted Value of Birth Weight}
%  \label{boeBW}
%  \begin{tabular}{lccccc}
%    \toprule
%    Age & $\frac{\partial wage}{\partial 250g}$ & Median Salary & Present Value & Participation Rate & PV$\times$ Particip \\ \midrule
%    \multicolumn{6}{l}{\textsc{Panel A: Females}} \\
%    25-30 & 2.11\% & \$27,473  & \$741.13 &73.8&\$546.95 \\
%    30-35 & 1.80\% & \$30,317  & \$546.66 &73.8&\$403.43 \\
%    35-40 & 0.76\% & \$31,580  & \$188.38 &74.1&\$139.59 \\
%    40-45 & 0.03\% & \$31,667  & \$5.84   &74.1&\$4.33   \\
%    45-50 & 1.31\% & \$31,810  & \$200.80 &73.8&\$148.18 \\
%    50-55 & 1.63\% & \$31,880  & \$196.19 &73.8&\$144.79 \\
%    55-60 & 1.52\% & \$30,183  & \$135.72 &58.8&\$79.80  \\ \midrule
%    TOTAL &        &           & \$2014.75 &     &\$1476.08\\ \midrule
%    \multicolumn{6}{l}{\textsc{Panel B: Males}} \\
%    25-30 & 1.25\% & \$34,328 & \$548.61 &  88.7 & \$486.61 \\
%    30-35 & 1.23\% & \$41,720  & \$514.05 & 88.7& \$455.96 \\
%    35-40 & 1.07\% & \$48,007  & \$403.18 & 90.5& \$364.88 \\
%    40-45 & 1.01\% & \$50,711  & \$314.98 & 90.5& \$285.06 \\
%    45-50 & 0.99\% & \$51,021  & \$243.39 & 85.6& \$208.34 \\
%    50-55 & 0.61\% & \$51,110  & \$117.71 & 85.6& \$100.76 \\
%    55-60 & 0.70\% & \$50,104  & \$103.75 & 76.8& \$79.68  \\ \midrule
%    TOTAL &        &           & \$2245.67 &     &\$1981.30 \\ \bottomrule
%    \multicolumn{6}{p{14.6cm}}{{\footnotesize \textsc{Notes}:  All estimates
%        in column 2 are provided by \citet{Bharadwajetal2015} (refer to their
%        table 4), and are per additional 250 grams of birth weight. Median
%        salary figures are provided by the \citet{USCB2016},
%        and labour participation rates are provided by the \citet{BLS2017}. All
%        present values are calculated as a deferred annuity, assuming that
%        the child will reach the labour market in 25 years, and a discount
%    rate of 5\%.}}
%  \end{tabular}
%\end{table}
%
\begin{landscape}
\begin{figure}[htpb!]
  \begin{center}
    \caption{Heterogeneity in Conjoint Analysis}
    \label{hetCA}
  \includegraphics[scale=0.84]{../results/DCE/Figures/parentalSubsets.eps}
  \end{center}
  \floatfoot{\textsc{Notes:} Methods are identical to those described in notes to figure \ref{DCE-samp}. The full sample is split by parents or non-parents (panels A and B), and then non-parents are split into those who report planning to have children (or being already pregnant) versus those who do not plan to have children (panels C and D).}
\end{figure}
\end{landscape}
\begin{landscape}
\input{./../results/DCE/Regressions/conjointGroups.tex}
\end{landscape}



%%% MALES: median 37,138               PV      Participation Rate
% 25-30 -- 250 grams = +1.25\% 34,328  548.607 88.7
% 30-35 -- 250 grams = +1.23\% 41,720  514.051 88.7
% 35-40 -- 250 grams = +1.07\% 48,007  403.179 90.5
% 40-45 -- 250 grams = +1.01\% 50,711  314.983 90.5
% 45-50 -- 250 grams = +0.99\% 51,021  243.389 85.6
% 50-55 -- 250 grams = +0.61\% 51,110  117.708 85.6
% 55-60 -- 250 grams = +0.70\% 50,104  103.752 76.8
%TOTAL = 2245.67
%TOTAL WITH PARTICIPATION = 1981.295

% FEMALES: median 23,769               PV
% 25-30 -- 250 grams = +2.11\% 27,473  741.125 73.8
% 30-35 -- 250 grams = +1.80\% 30,317  546.657 73.8
% 35-40 -- 250 grams = +0.76\% 31,580  188.380 74.1
% 40-45 -- 250 grams = +0.03\% 31,667  5.842   74.1
% 45-50 -- 250 grams = +1.31\% 31,810  200.795 73.8
% 50-55 -- 250 grams = +1.63\% 31,880  196.190 73.8
% 55-60 -- 250 grams = +1.52\% 30,183  135.716 58.8
%TOTAL = 2014.75
%TOTAL WITH PARTICIPATION = 1476.078


\clearpage
\setcounter{table}{0}
\renewcommand{\thetable}{A\arabic{table}}
\setcounter{figure}{0}
\renewcommand{\thefigure}{A\arabic{figure}}

\appendix
\section{Appendix Figures and Tables}
\begin{figure}[htpb!]
  \begin{center}
    \caption{Birth Weight from Administrative Data}
    \label{bwt-nvss}
  \includegraphics[scale=0.9]{../results/births/birthweight.eps}
  \end{center}
  \floatfoot{\textsc{Notes:} Full birth weight distribution from all US births occurring in 2013 observed from NVSS birth certificate data (values below 500 grams or above 5000 grams are removed for display purposes). 84.09\% of all births fall in the ``normal'' birth range of 2500 to 4000g.  Of non-normal birth weights, 8.02\% are low birth weight ($<$ 2,500 grams), and the remaining 7.89\% were large ($>$ 4,000 grams).}
\end{figure}


\begin{figure}[htpb!]
  \begin{center}
    \caption{Discrete Choice Experimental Results with Categorical Costs (Main Sample)}
    \label{DCE-full-samp}
  \includegraphics[scale=0.9]{../results/DCE/Figures/Conjoint_Sample.eps}
  \end{center}
  \floatfoot{\textsc{Notes:} Refer to note to figure \ref{DCE-samp}.  This figure is based
    on an identical sample, however now using categorical rather than a linear measure of
    costs.}
\end{figure}

\begin{figure}[htpb!]
  \begin{center}
    \caption{Discrete Choice Experimental Results (Full Sample)}
    \label{DCE-sampFull}
  \includegraphics[scale=0.9]{../results/DCE/Figures/Conjoint_All_continuous.eps}
  \end{center}
  \floatfoot{\textsc{Notes:} Refer to notes to figure \ref{DCE-samp}.  This figure is
    identical, however now also including the $\sim 5\%$ removed for failing consistency
    checks.}
\end{figure}


%\begin{figure}[htpb!]
%  \begin{center}
%    \caption{Discrete Choice Experimental Results with Categorical Costs (Full Sample)}
%    \label{DCE-full-all}
%  \includegraphics[scale=0.9]{../results/DCE/Figures/Conjoint_All.eps}
%  \end{center}
%  \floatfoot{\textsc{Notes:} }
%\end{figure}

\input{./../results/DCE/Regressions/conjointGroups-wts.tex}
\begin{landscape}
\input{./../results/DCE/Regressions/conjointGender.tex}
\end{landscape}
\begin{landscape}
\input{./../results/DCE/Regressions/conjointWTP-interactions.tex}
\end{landscape}



\begin{figure}[htpb!]
  \begin{center}
    \caption{Discrete Choice Experiment Framing}
    \label{DCE-frame}
  \includegraphics[scale=0.7]{surveyQs/Q1.png}
  \end{center}
  %\floatfoot{\textsc{Notes:} }
\end{figure}

\begin{figure}[htpb!]
  \begin{center}
    \caption{Discrete Choice Experiment Options}
    \label{DCE-options}
  \includegraphics[scale=0.7]{surveyQs/Q2.png}
  \end{center}
  %\floatfoot{\textsc{Notes:} }
\end{figure}

\begin{figure}[htpb!]
  \begin{center}
    \caption{Discrete Choice Experiment Example}
    \label{DCE-example}
  \includegraphics[scale=0.7]{surveyQs/Q3.png}
  \end{center}
  %\floatfoot{\textsc{Notes:} }
\end{figure}


%\begin{figure}[htpb!]
%  \begin{center}
%    \caption{Examining the Necessity for Round Fixed Effects}
%    \label{DCE-asm2}
%  \includegraphics[scale=0.81]{../results/DCE/Figures/roundOrder.eps}
%  \end{center}
%  \floatfoot{\textsc{Notes:} Each estimate and confidence interval refers to
%    the estimated likelihood of choosing a particular profile given an increase
%    in birth weight by the round order of the experiment. All confidence intervals
%    are clustered by respondent.}
%\end{figure}
%
%\begin{figure}[htpb!]
%  \begin{center}
%    \caption{Examining the Necessity for Position Order Fixed Effects}
%    \label{DCE-asm3}
%  \includegraphics[scale=0.81]{../results/DCE/Figures/attributeOrder.eps}
%  \end{center}
%  \floatfoot{\textsc{Notes:} Each estimate and confidence interval refers to
%    the estimated likelihood of choosing a particular profile given an increase
%    in birth weight by the position of birth order in the list of characteristics
%    in the profile (refer to figure \ref{DCE-example} for an example where
%    birth weight is located in position 3). All confidence intervals are
%    clustered by respondent.}
%\end{figure}

\begin{landscape}
\begin{figure}[htpb!]
  \begin{center}
    \caption{Mechanical Turk Front Page}
    \label{MTurkAdd}
  \includegraphics[scale=0.65]{surveyQs/MTurkScreen.png}
  \end{center}
  \floatfoot{\textsc{Notes:} Respondents first see the survey front page on MTurk before being redirected to the survey located Qualtrics (as displayed in figures \ref{DCE-frame}, \ref{DCE-options} and \ref{DCE-example}). A description of the process followed by respondents is provided in appendix \ref{app:procedure}.}
\end{figure}
\end{landscape}

\begin{table}
  \caption{Comparison of Confidence Intervals for WTP from Delta Method and Block Bootstrap}
  \label{bootstrapCI}
  \begin{tabular}{lcccc} \toprule
    Estimation & WTP Point & Delta Method & Block Bootstrap & Full \\
    Sample     &  Estimate & 95\% CI      & 95\% CI         & Results \\ \midrule
    Main Sample&  \$1438.3 & [\$1119.4;\$1757.1] & [\$1107.2;\$1769.3] & Table \ref{WTPreg} \\
    Parents    &  \$1718.4 & [\$1232.4;\$2204.4] & [\$1233.9;\$2202.8] & Table \ref{WTPgreg} \\
    Non-Parents&  \$1172.4 & [\$753.0;\$1591.8]  & [\$757.5;\$1587.3]  & Table \ref{WTPgreg} \\
    Women      &  \$1275.7 & [\$856.4;\$1695.0]  & [\$864.9;\$1686.6]  & Table \ref{WTPgreg} \\
    Men        &  \$1663.8 & [\$1173.7;\$2153.8] & [\$1173.3;\$2154.2] & Table \ref{WTPgreg} \\
   Girl Profile&  \$1398.3 & [\$974.6;\$1821.9]  & [\$947.8;\$1848.7]  & Table \ref{WTPgend} \\
    Boy Profile&  \$1476.7 & [\$1073.1;\$1880.3] & [\$1059.7;\$1893.7] & Table \ref{WTPgend} \\
    \midrule
    \multicolumn{5}{p{13.6cm}}{{\footnotesize \textsc{Notes}: WTP point estimates
        are all calculated as the ratio of the coefficient on birth weight (in
        1000s of grams) to the coefficient on costs in dollars. The delta method
        for the 95\% confidence interval is displayed in tables throughout the
        paper and is calculated from directly from regression coefficients and
        the maximum likelihood function, while the block bootstrap confidence
        interval is based on re-sampling with replacement over survey respondents
        (not over individual profiles) in order to maintain the correct dependence
        structure within survey respondents.  In all cases, 1,000 bootstrap
        samples are performed, and the 95\% confidence interval is taken using the
        WTP in each of these 1,000 replications. Additional discussion of the
    relative merits of each method is available in \citet{Hole2007}.}}
  \end{tabular}
\end{table}

\clearpage

\begingroup
\setlength{\LTleft}{-20cm plus -1fill}
\setlength{\LTright}{\LTleft}
\begin{longtable}{lccc}
  \caption{Geographical Coverage} \label{tab:cover} \\
  \hline
  State Name & Percent & Percent & Difference \\
             & MTurk      & Census Bureau & (\%)          \\ \hline \endhead
  \input{./../results/DCE/Summary/GeographicCoverage.tex}
  \hline
  \multicolumn{4}{p{10.4cm}}{{\footnotesize\textsc{Notes:} Columns present the percent of respondents from the MTurk sample, the percent of residents according to US Census Bureau records (2015), and the difference between the percent of MTurk respondents and residents.}}
\end{longtable}
\endgroup

%\begin{landscape}
%\begin{figure}[htpb!]
%  \begin{center}
%    \caption{Willingness to Pay by Gender of Index Child}
%    \label{graphGend}
%  \includegraphics[scale=1.4]{../results/DCE/Figures/WTP_relative_gends.eps}
%  \end{center}
%  \floatfoot{\textsc{Notes:} Thick lines present point estimates of the marginal WTP for birth weight for each gender (based on the minimum value of 2,500 grams for each gender).  95\% confidence intervals for each WTP are based on the delta method, and allow for arbitrary correlation of choices within each survey respondent.}
%\end{figure}
%\end{landscape}




\section{Survey Response Procedure}
\label{app:procedure}
Below we describe the survey response procedure as seen by survey respondents.
\begin{enumerate}
\item All respondents meeting survey criteria ($>95\%$ approval rating, $>100$ completed MTurk tasks, US based, and non-participants in the pilot) were able to see the Mechanical Turk HIT with the title ``Link to Survey'' along with the description displayed in appendix figure \ref{MTurkAdd}. Respondents are instructed that payment is conditional upon completing the survey and providing a randomized code which is displayed at the end of the survey.
\item Respondents accept participation and are directed to the discrete choice experiment on the Qualtrics survey platform.
\item Respondents must complete each question in order to move forward, and after completing the survey the randomized code is displayed.
\item Respondents return to the MTurk front page, enter their unique completed survey code and receive payment.
\end{enumerate}


\end{document}


In order to examine the
stability of this crude calculation to a more flexible set of estimates,
we turn to the results of \citet{Bharadwajetal2015}, who are able to estimate
the heterogeneous returns to birth weight over the entire labour market
portion of the life cycle. Although these estimates are based on Swedish
rather than US data, they provide a rate of return, and they are the only
estimates of which we are aware that allow for temporal variations
in returns.\footnote{What's more, at least on average, the estimates
  from \citet{Bharadwajetal2015} are largely in agreement with the
  other (static) values outlined in table \ref{litrev}.}

These calculations also allow us to take account of the fact that
the estimated returns to birth weight are different for boys and
girls.  This is discussed extensively by \citet{RosenzweigZhang2013}
who offer a ``skill versus brawn'' explanation of the effects of
additional \emph{in-utero} nutrition on returns to human capital,
suggesting that educational returns of birth weight are larger for
girls, and labour market returns of birth weight are larger for boys.
In our experimental data, we do observe a differential WTP for birth weight
when the experimental profile is a boy versus when it is a girl. In
table \ref{WTPgend} we estimate a slightly higher willingness to pay
for birth weight in boy children (\$1.47 per gram versus \$1.40 per gram).
Although this difference is not statistically distinguishable at typical
levels, in examining non-parametric categories of birth weight, we
observe that the WTP distribution for all weights of boys dominates
the same distribution for all weights of girls (see columns 2 and 4,
and appendix figure \ref{graphGend}).

In table \ref{boeBW} we calculate the present value of a gram of
birth weight separately for each gender using prevailing US wage
rates, labour market participation\footnote{In the sense that parents
  form expectations over future labour market outcomes using current
  conditions, these values will capture the most complete set of
  information available at the survey date.}, and the labour market
returns to birth weight estimated by \citet{Bharadwajetal2015}.  This
consists of discounting to present day the full expected income
flows for a median individual who enters the labour market at 25 years
of age, and works up until the age of 60.  In this case, even when
considering life cycle variations in labour market conditions and the
effect of birth weight, we still find that the returns to birth weight
exceed parental willingness to pay.  For girls, the crude estimate of
the return to an additional 250 grams of weight is \$1476.08, or
approximately \$6 per gram, and for boys these values are \$1981.30,
and \$8 per gram respectively.  These values, while closer to our
estimates of WTP, still are approximately 4-5 times that experimentally
elicited values.






%%%%%%%%%%%%%%%%%%%%%%%%%%%%%%%%%%%%%%%%%%%%%%%%%%%%%%%%%%%%%%%%%%%%%%%%%%%%
SOME REFS
%%%%%%%%%%%%%%%%%%%%%%%%%%%%%%%%%%%%%%%%%%%%%%%%%%%%%%%%%%%%%%%%%%%%%%%%%%%%


\citet{Royer2009}: ``Consistent with previous studies, I estimate a statistically significant relationship
between birth weight and long-run and intergenerational outcomes. In particular,
the heavier twin obtains more education, gives birth to heavier children, and has
fewer pregnancy complications. In sharp contrast to earlier research, however, these
effects tend to be quite small with the exception of pregnancy complications. For a
200 gram increase in birth weight, which likely is an achievable policy manipula-
tion, education would be projected to rise by roughly 0.04 of one year. These nega-
tive effects of birth weight do not appear to be persistent across generations, as the
estimated intergenerational correlation in birth weight is only 0.07. In contrast to
other studies (e.g., Black, Devereux, and Salvanes 2007), .''

\citet{BehrmanRosenzweig2004}: ``Our estimates provide a number of clear results. First,
they indicate that increasing fetal growth has a significant positive effect on schooling
attainment that is underestimated by 50\% if there is no control for genetic and family
background endowments as in cross-sectional estimates.
Second, our estimates indicate that intrauterine nutrient
consumption does not have any persistent effects on adult
BMI---increasing birth weight is not a cause of adult obesity.
Third, our results indicate that the heritable component of
birth weight plays the dominant role in the intergenerational
correlation of birth weights. Fourth, the estimates indicate
that intrauterine nutrient intake significantly affects adult
height, consistent with the literature that makes use of
height statistics to gauge childhood nutritional investments
over time and across countries. Fifth, although we
find evidence that augmenting birth weight, particularly
among lower-birth weight babies, would have significant
labor-market payoffs, we show that the strong cross-country
correlation between incomes and birth weight substantially
overstates the reduction in world earnings inequality that
would arise from reducing cross-country disparities in birth-weights.''

Finally, the within-MZ point estimate of the reduced-
form effect of fetal growth on wages indicates that augment-
ing a child's birth weight by a 1 lb. increases her adult
earnings by over 7\%. That increasing fetal growth and
birth weight increases earnings is not surprising given the
estimates suggesting that increasing birth weight increases
schooling.

\citet{Blacketal2007}: ``We find that birth weight does matter. Consistent with earlier
work, we find that twin fixed effects estimates of the effect of birth weight on
short-run outcomes such as one-year infant mortality are much smaller than their
cross-sectional equivalents. However, studying only short-run outcomes may lead to
incorrect inferences about the longer-run effects of birth weight; we find
that birth weight has a significant effect on longer-run outcomes
such as height, IQ at age 18, earnings, and education,
and the fixed effects estimates are similar in size to cross-sectional ones.''

FOOTNOTE 13 (p.\ 416) It is interesting to note that the LBW indicator fits
most poorly for all outcomes. This suggests
that using cutoffs such as $<$2,500
grams as the variable of interest may not be appropriate for this type of
analysis. We have also tried an indicator for LBW ($<$2,500) in the
same specifications.  The continuous measure dominates
for all outcomes and the effect of LBW is always statistically
insignificant and often has the wrong sign. In
the same vein, we have also tried including both ln(birth weight)
and birth length; with the exception of
of height at age eighteen, birth length is always dominated
by ln(birth weight).



\citet{Oreopoulosetal2008}: ``We find evidence of longer-term
consequences of infant health both across families, within siblings, and
within twin pairs, although different measures of infant health predict outcomes
differently. The
results suggest strong effects of infant health on death between ages one and 17,
grade completion, and months on social assistance after age 18, even for ranges
not considered overtly concerning (for example, birth weights between 2,500 and
3,500 grams and Apgar scores of seven or eight)''

\citet{Figlioetal2014} We make use of a new data resource -- merged birth and school records for all children born in Florida from 1992 to 2002 -- to study the relationship between birth weight and cognitive development. Using singletons as well as twin and sibling fixed effects models, we find that the effects of early health on cognitive development are essentially constant through the school career; that these effects are similar across a wide range of family backgrounds; and that they are invariant to measures of school quality. We conclude that the effects of early health on adult outcomes are therefore set very early.

\citet{ConleyBennet2000}: ``The results, presented in Table 2, indicate
that low birth weight negatively affects educational progress, even
after factoring out family-specific conditions.''

\citet{Bharadwajetal2015} In line with Black, Devereux, and Salvanes (2007)
and others who have examined the relationship between birth weight
and income in early adulthood, we find birth weight to have
a positive and statistically significant effect on income. We find positive
effects on both permanent income as well as on income over large
parts of the life cycle. After age 50, however, the effect seems
to fade out. Interestingly, we find that the impact of birth weight on income
has remained rather constant across cohorts born
almost 50 years apart.

\citet{CookFletcher2015} For Panel B, we estimate our proposed interaction
model, replacing IQ with wages later in life. The estimated associations of
neuroplasticity, birth weight, and the interaction between the two
variables, however, are similar to the regressions with IQ. Birth
weight has a positive and statistically significant association with
the wage rate, but the magnitude of this effect is dependent upon
our measure for neuroplasticity. When considering the sibling fixed
effects estimation of column (3), a 10\% increase in birth weight is
associated with roughly a 15\% increase in wage for the least plastic
individuals. For the median, neuroplasticity score, a 10\% increase
in birth weight is associated with roughly a 5\% increase in wage.
And for the maximum neuroplasticity score, a 10\% increase in birth
weight is negatively associated with wage. As with IQ, neuroplasticity
moderates the associations between birth weight and later life outcomes

\citet{Fletcher2011}: not wages.

OTHER METHODS
The first involves exploiting historical events (e.g., the 1944 Dutch famine
(Lumey and Stein 1997) and the 1918 influenza epidemic (Almond 2006)), which
altered the in utero environment through starvation, stress, and/or sickness.

and Phil Oreopoulos et al. 2006.


\citet{JohnsonSchoeni2011}
models of cognition, childhood health, education, and health, earnings,
and wages in adulthood we obtain a much more compelling understanding of the
long-run effects of early life events. Indeed, our findings on the effects of early-
life events using the national PSID sample are remarkably consistent with a small
but growing set of very recent studies by economists (Black, Devereux, and Salvanes,
2007; Oreopoulos et al., 2008;
Almond and Mazumder, 2005; Almond and Chay, 2006).

\citet{CurrieMoretti2007}
Low birth weight has been used as the leading indicator of poor health
among  newborns for many years.  In 1996,  the infant mortality rate for
babies over 2,500 grams was 2.77 compared to 17.45 for babies between
1,500 and 2,500 grams, and 259.35 for babies less than 1,500 grams (Conley
and Bennett 2001). Follow ups indicate that low birth weight babies have
lower scores on a variety of tests of intellectual and social development
(Breslau et al. 1994; Brooks-Gunn, Klebanov, and Duncan 1996). \textbf{Currie
and  Hyson  (1996)}  find  that  low  birth  weight  was  predictive  of  lower
schooling attainments, earnings, and employment probabilities as of age
33,  regardless  of  the  parents'  socioeconomic  status.

\citet{CurrieHyson1999}



ADD DCC:

Discuss why we do not use papers such as \citet{Almond2006} (no direct
estimates of returns to birth weight).  Also do not use papers like
\citet{Figlioetal2014} and others from their footnote 1 as they are
too short term.

Almond, Chay, and Lee (2005) and Conley, Strully, and Bennett (2003)
on neonatal outcomes and hospital costs
Torche and Echevarria (2011) on fourth-grade mathematics test scores.
Bharadwaj, Eberhard, and Neilson (2013), in a current working paper,
study fourth-grade test scores and grades in school (also in Chile)








